\documentclass[letterpaper,notitlepage,twoside]{article}

% Basic imports, increase margins...
\usepackage[margin=0.75in]{geometry}
\usepackage{amssymb}
\usepackage{amsmath}

% Finite State Machine stuff
\usepackage{pgf}
\usepackage{tikz}
\usetikzlibrary{arrows,automata}

% Format tables nicely
\usepackage[latin1]{inputenc}
\usepackage{array}
\usepackage{booktabs}
\setlength{\heavyrulewidth}{1.5pt}
\setlength{\abovetopsep}{4pt}

\usepackage{amsfonts} 
\usepackage{amssymb}
\usepackage{amsmath,amsthm}

\renewcommand{\implies}{\Rightarrow} % redefine command "implies"  
\renewcommand{\iff}{\Leftrightarrow} % double arrow
\newcommand{\maps}{\rightarrow} % define command "map" 
\newcommand{\union}{\cup}
\newcommand{\intersect}{\cap}
\newcommand{\N}{\mathbb{N}} % natural number 
\newcommand{\Q}{\mathbb{Q}} % rational number 
\newcommand{\R}{\mathbb{R}} % real number 
\newcommand{\Z}{\mathbb{Z}} % integers 
\newcommand\tab[1][1cm]{\hspace*{#1}} %\tab command

% Add more packages that you use here...

\begin{document}
\title{Homework 14}
\author{Joe Baker, Brett Schreiber, Brian Knotten}
\maketitle

\section*{21}

\subsection*{a}
Consider that every Boolean circuit can be represented by AND, OR, and NOT gates. \\
Consider a Boolean circuit $C$ containing $n$ gates. There exists a corresponding Boolean formula $F$ of size $n$, because AND, OR, and NOT gates can be represented in a Boolean formula as logical operators $\land$'s, $\lor$'s, and $\neg$'s. Since there is a one-to-one mapping of gates to logical operators, there exists a Boolean formula $F$ with an output equivalent to $C$ of size $n$.

Any Boolean function $F$ for an input size $n$ can be encoded into a string $x$ of size $|x| = S = 2^n$, where each bit represents an output for a given input. For example, if $n = 4$, the first bit of $x$, either a 0 or 1, represents the output for the input 0000. The second bit of $x$ represents the output for input 0001... and the last bit of $x$ represents the output for input 1111.

The formula for $F$ can be naively implemented through the following algorithm:

List all binary numbers of length $n$ from $0$ to $2^n$. Let $B$ be this list.
For each number $b$ in $B$: \\
\tab for each bit $b_i$ in $b$: \\
\tab\tab if $b_i = 0$, write out $\neg x_i$ \\
\tab\tab if $b_i = 1$, write out $x_i$ \\
\tab Join together all of these boolean expressions with $\land$'s \\
Join together all of these boolean expressions with $\lor$'s \\

Here is an example Boolean function $f$ encoded as 1011. This means $f$ has the following outputs: \\
$f(00) = 1$ \\
$f(01) = 0$ \\
$f(10) = 1$ \\
$f(11) = 1$ \\

The corresponding boolean formula using the naive implementation is:
$f(x_1x_2) = (\neg x_1 \land \neg x_2) \lor \\
\neg(\neg x_1 \land x_2) \lor \\
(x_1 \land \neg x_2) \lor \\
(x_1 \land x_2)$ \\

So every boolean function on $n$ bits has a formula of size at most $2^n$.
Since $\land$, $\lor$, and $\neg$ exist as gates in circuits, this formula can be converted into a circuit with size at most $2^n$ gates.

\subsection*{b}
First, show that $f$ computed by circuit $|C| = S \rightarrow$ $f$ computed by $S$-line program: \\
For each Boolean gate (NOT, AND, OR) in $C$, there exists an equivalent straight-line program statement with a left-side assignment variable corresponding to the output of the gate and a right side consisting of either a boolean operation ($\land$, $\lor$) performing the gate's operation on input variables corresponding to the input of the gate, or the negation ($\neg$) of an input variable corresponding to the input of the gate. \\
Therefore, given a circuit $|C| = S$, we can construct an equivalent straight-line program with $S$ statements by iterating over each gate in $C$ and converting it to an equivalent straight-line program statement. \\ \\ \\
Next, show that $f$ computed by $S$-line program $\rightarrow$ $f$ computed by circuit $|C| = S$: \\
For every straight-line program statement there exists an equivalent Boolean gate (NOT, AND, OR) with inputs corresponding to the right-side variable(s) of the statement that performs the corresponding Boolean operation ($\neg$, $\land$, $\lor$) of the right-side of the statement and an output corresponding to the left-side assignment variable of the statement. \\
Therefore, given a straight-line program with $S$ statements, we can construct an equivalent circuit $|C| = S$ by iterating over each statement and converting it to an equivalent Boolean gate. \\

\section*{22}

\end{document}
