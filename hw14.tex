\documentclass[letterpaper,notitlepage,twoside]{article}

% Basic imports, increase margins...
\usepackage[margin=0.75in]{geometry}
\usepackage{amssymb}
\usepackage{amsmath}

% Finite State Machine stuff
\usepackage{pgf}
\usepackage{tikz}
\usetikzlibrary{arrows,automata}

% Format tables nicely
\usepackage[latin1]{inputenc}
\usepackage{array}
\usepackage{booktabs}
\setlength{\heavyrulewidth}{1.5pt}
\setlength{\abovetopsep}{4pt}

\usepackage{amsfonts} 
\usepackage{amssymb}
\usepackage{amsmath,amsthm}

\renewcommand{\implies}{\Rightarrow} % redefine command "implies"  
\renewcommand{\iff}{\Leftrightarrow} % double arrow
\newcommand{\maps}{\rightarrow} % define command "map" 
\newcommand{\union}{\cup}
\newcommand{\intersect}{\cap}
\newcommand{\N}{\mathbb{N}} % natural number 
\newcommand{\Q}{\mathbb{Q}} % rational number 
\newcommand{\R}{\mathbb{R}} % real number 
\newcommand{\Z}{\mathbb{Z}} % integers 
\newcommand\tab[1][1cm]{\hspace*{#1}} %\tab command

% Add more packages that you use here...

\begin{document}
\title{Homework 14}
\author{Joe Baker, Brett Schreiber, Brian Knotten}
\maketitle

\section*{21}

\subsection*{a}
Any Boolean function $F$ for an input size $n$ can be encoded into a string $s$ of size $2^n$, where each bit represents an output for a given input. For example, if $n = 4$, the first bit of $s$, either a 0 or 1, represents the output for the input 0000. The second bit of $s$ represents the output for input 0001... and the last bit of $s$ represents the output for input 1111.

The formula for $F$ can be naively implemented through the following algorithm:

List all binary numbers of length $n$ from $0$ to $2^n$. Let $B$ be this list.
For each number $b$ in $B$: \\
\tab for each bit $b_i$ in $b$: \\
\tab\tab if b_i = 0, write out $\not x_i$ \\
\tab\tab if b_i = 1, write out $x_i$ \\
\tab Join together all of these boolean expressions with $\land$s \\
Join together all of these boolean expressions with $\lor$s \\

Here is an example Boolean function $f$ encoded as 1011. This means $f$ has the following outputs: \\
$f(00) = 1$ \\
$f(01) = 0$ \\
$f(10) = 1$ \\
$f(11) = 1$ \\

The corresponding boolean formula using the naive implementation is:
$f(x_1x_2) = (\neg x_1 \land \neg x_2) \lor \\
\neg(\neg x_1 \land x_2) \lor \\
(x_1 \land \neg x_2) \lor \\
(x_1 \land x_2)$ \\

So every boolean function on $n$ bits has a formula of size at most $2^n$.
Since $\land$, $\lor$, and $\neg$ exist as gates in circuits, this formula can be converted into a circuit with size at most $2^n$ gates.

\subsection*{b} For every boolean function $f$ can be computed by a Boolean circuit of size $S$ iff $f$ can be computed by a straight-line program with $S$ lines. \\
First, show that $f$ computed by circuit $|C| = S \rightarrow$ $f$ computed by $S$-line program: \\
For each Boolean gate in $C$, there exists an equivalent straight-line program statement with a left-side assignment variable corresponding to the output of the gate and a right side consisting of either a boolean operation performing the gate's operation on input variables corresponding to the input of the gate, or the negation of an input variable corresponding to the input of the gate. \\
Next, show that $f$ computed by $S$-line program $\rightarrow$ $f$ computed by circuit $|C| = S$: \\
For every straight-line program statement there exists an equivalent Boolean gate with inputs corresponding to the right-side variable(s) of the statement

\section*{22}

\end{document}
