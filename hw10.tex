\documentclass[letterpaper,notitlepage,twoside]{article}

% Basic imports, increase margins...
\usepackage[margin=0.75in]{geometry}
\usepackage{amssymb}
\usepackage{amsmath}

% Finite State Machine stuff
\usepackage{pgf}
\usepackage{tikz}
\usetikzlibrary{arrows,automata}

% Format tables nicely
\usepackage[latin1]{inputenc}
\usepackage{array}
\usepackage{booktabs}
\setlength{\heavyrulewidth}{1.5pt}
\setlength{\abovetopsep}{4pt}

\usepackage{amsfonts} 
\usepackage{amssymb}
\usepackage{amsmath,amsthm}

\renewcommand{\implies}{\Rightarrow} % redefine command "implies"  
\renewcommand{\iff}{\Leftrightarrow} % double arrow
\newcommand{\maps}{\rightarrow} % define command "map" 
\newcommand{\union}{\cup}
\newcommand{\intersect}{\cap}
\newcommand{\N}{\mathbb{N}} % natural number 
\newcommand{\Q}{\mathbb{Q}} % rational number 
\newcommand{\R}{\mathbb{R}} % real number 
\newcommand{\Z}{\mathbb{Z}} % integers 
\newcommand\tab[1][1cm]{\hspace*{#1}} %\tab command

% Add more packages that you use here...

\begin{document}
\title{Homework 10}
\author{Joe Baker, Brett Schreiber, Brian Knotten}
\maketitle

\section*{14}
Consider the language $L = \{<M> \mid <M> $ is a Turing Machine that loops forever$\}$.
One of the strings in $L$ can effectively written as a valid Java program containing the line: \\\\

while(true) \{\} \\

Mini Java's looping mechanism must be given a finite number describing how many times to loop. So all Mini Java programs cannot simulate an infinite loop. Therefore, $L$ is a language that Java can simulate, but Mini Java cannot.

\section*{15}
\subsection*{a}
A TM $M$ can be defined as follows so that $L(M) = A$: \\
On input $x$: \\
\tab Instantiate $c = 0$ on the working tape. \\
\tab for each character $x_i \in x$: \\
\tab\tab if $x_i =$ '(', then increment $c$. \\
\tab\tab else, if $x_i =$ ')', then decrement $c$. \\
\tab\tab if $c < 0$ reject. (There is a right paren before a left one). \\
\tab Accept if an only if the final value of $c = 0$. \\
$L(M) = A$ because $L(M)$ only accepts when the number of left parentheses matches the number of right parentheses. \\
$M$ runs in logspace because the in the worst case, the input $x$ to $M$ will be $n$ number of left parentheses. So the working tape has to count up to $n$. But by using the standard base-2 binary encoding of $n$, the working tape will only use a maximum of $\log(n)$ cells. \\

\subsection*{b}
A TM $N$ can be defined as follows so that $L(N) = B$: \\
Assume $N$ has two work tapes. \\
On input $x$: \\
\tab First pass: For each character $x_i, x_{i + 1}$ in $x$: \\
\tab\tab If $x_i, x_{i + 1} = $ '(', ']', reject, \\
\tab\tab or if $x_i, x_{i + 1} = $ '[', ')', reject. \\

Set $c_1 = 0$ on the first working tape. \\
Set $c_2 = 0$ on the second working tape. \\
\tab Second pass: For each character $x_i$ in $x$: \\
\tab\tab If $x = $ '(', increment $c_1$ \\
\tab\tab or if $x = $ ')', decrement $c_1$ \\
\tab\tab or if $x = $ '[', increment $c_2$ \\ 
\tab\tab or if $x = $ ']', decrement $c_2$ \\
\tab\tab If $c_1 < 0$ or $c_2 < 0$, then immediately reject. \\
\tab If $c_1 = 0$ and $c_2 = 0$, then accept. Otherwise, reject. \\

$N$ runs in logspace, even though it uses two work tapes, since $2\log(n) = O(\log(n))$.
\end{document}
