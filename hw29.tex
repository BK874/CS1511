\documentclass[letterpaper,notitlepage,twoside]{article}

% Basic imports, increase margins...
\usepackage[margin=0.75in]{geometry}
\usepackage{amssymb}
\usepackage{amsmath}

% Finite State Machine stuff
\usepackage{pgf}
\usepackage{tikz}
\usetikzlibrary{arrows,automata}

% Format tables nicely
\usepackage[latin1]{inputenc}
\usepackage{array}
\usepackage{booktabs}
\setlength{\heavyrulewidth}{1.5pt}
\setlength{\abovetopsep}{4pt}

\usepackage{amsfonts} 
\usepackage{amssymb}
\usepackage{amsmath,amsthm}

\renewcommand{\implies}{\Rightarrow} % redefine command "implies"  
\renewcommand{\iff}{\Leftrightarrow} % double arrow
\newcommand{\maps}{\rightarrow} % define command "map" 
\newcommand{\union}{\cup}
\newcommand{\intersect}{\cap}
\newcommand{\N}{\mathbb{N}} % natural number 
\newcommand{\Q}{\mathbb{Q}} % rational number 
\newcommand{\R}{\mathbb{R}} % real number 
\newcommand{\Z}{\mathbb{Z}} % integers 
\newcommand\tab[1][1cm]{\hspace*{#1}} %\tab command

% Add more packages that you use here...
\usepackage{braket}

\begin{document}
\title{Homework 28}
\author{Joe Baker, Brett Schreiber, Brian Knotten}
\maketitle

\section*{53}
\subsection*{a}
Alice will do one of the following operations on $a$ depending on the values of $x$ and $y$.

If $x = 0$ and $y = 0$, leave $a$ as is, producing $\frac{1}{\sqrt 2} \ket{0} + \frac{1}{\sqrt 2} \ket{1}$ \\
If $x = 0$ and $y = 1$, rotate $a$ by $\frac{\pi}{2}$, producing $-\frac{1}{\sqrt 2} \ket{0} + \frac{1}{\sqrt 2} \ket{1}$ \\
If $x = 1$ and $y = 0$, rotate $a$ by $-\frac{\pa}{2}$, producing $\frac{1}{\sqrt 2} \ket{0} - \frac{1}{\sqrt 2} \ket{1}$  \\
If $x = 1$ and $y = 1$, rotate $a$ by $\frac{3\pi}{2}$, producing $-\frac{1}{\sqrt 2} \ket{0} - \frac{1}{\sqrt 2} \ket{1}$  \\

So these operations will put $a$ into one of the four Bell states.

\subsection*{b}
If $x = 0$ and $y = 0$, then  $a = \frac{1}{\sqrt 2} \ket{0} + \frac{1}{\sqrt 2} \ket{1}$ and $b = \frac{1}{\sqrt 2} \ket{0} + \frac{1}{\sqrt 2} \ket{1}$ \\
If $x = 0$ and $y = 1$, then $a = -\frac{1}{\sqrt 2} \ket{0} + \frac{1}{\sqrt 2} \ket{1}$ and $b = \frac{1}{\sqrt 2} \ket{0} + \frac{1}{\sqrt 2} \ket{1}$ \\
If $x = 1$ and $y = 0$, then $a = \frac{1}{\sqrt 2} \ket{0} - \frac{1}{\sqrt 2} \ket{1}$ and $b = \frac{1}{\sqrt 2} \ket{0} + \frac{1}{\sqrt 2} \ket{1}$ \\
If $x = 1$ and $y = 1$, then $a =\frac{1}{\sqrt 2} \ket{0} - \frac{1}{\sqrt 2} \ket{1}$ and $b = \frac{1}{\sqrt 2} \ket{0} + \frac{1}{\sqrt 2} \ket{1}$ \\

\subsection*{c}
Bob wil perform a Bell measurment on $a$, $H(a)$. There are four possible outcomes depending on what $a$ is. \\
When $a = \frac{1}{\sqrt 2} \ket{0} + \frac{1}{\sqrt 2} \ket{1}$:\\
$H(\frac{1}{\sqrt 2} \ket{0} + \frac{1}{\sqrt 2} \ket{1}) = \frac{1}{2}(\ket{0} + \ket{1}) + \frac{1}{2}(\ket{0} - \ket{1}) = \ket{0}$\\\\

When $a = \frac{1}{\sqrt 2} \ket{0} - \frac{1}{\sqrt 2} \ket{1}$:\\
$H(\frac{1}{\sqrt 2} \ket{0} - \frac{1}{\sqrt 2} \ket{1}) = \frac{1}{2}(\ket{0} + \ket{1}) - \frac{1}{2}(\ket{0} - \ket{1}) = \ket{1}$\\\\

When $a = -\frac{1}{\sqrt 2} \ket{0} + \frac{1}{\sqrt 2} \ket{1}$:\\


When $a = -\frac{1}{\sqrt 2} \ket{0} - \frac{1}{\sqrt 2} \ket{1}$:\\

\end{document}
