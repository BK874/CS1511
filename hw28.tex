\documentclass[letterpaper,notitlepage,twoside]{article}

% Basic imports, increase margins...
\usepackage[margin=0.75in]{geometry}
\usepackage{amssymb}
\usepackage{amsmath}

% Finite State Machine stuff
\usepackage{pgf}
\usepackage{tikz}
\usetikzlibrary{arrows,automata}

% Format tables nicely
\usepackage[latin1]{inputenc}
\usepackage{array}
\usepackage{booktabs}
\setlength{\heavyrulewidth}{1.5pt}
\setlength{\abovetopsep}{4pt}

\usepackage{amsfonts} 
\usepackage{amssymb}
\usepackage{amsmath,amsthm}

\renewcommand{\implies}{\Rightarrow} % redefine command "implies"  
\renewcommand{\iff}{\Leftrightarrow} % double arrow
\newcommand{\maps}{\rightarrow} % define command "map" 
\newcommand{\union}{\cup}
\newcommand{\intersect}{\cap}
\newcommand{\N}{\mathbb{N}} % natural number 
\newcommand{\Q}{\mathbb{Q}} % rational number 
\newcommand{\R}{\mathbb{R}} % real number 
\newcommand{\Z}{\mathbb{Z}} % integers 
\newcommand\tab[1][1cm]{\hspace*{#1}} %\tab command

% Add more packages that you use here...
\usepackage{braket}

\begin{document}
\title{Homework 28}
\author{Joe Baker, Brett Schreiber, Brian Knotten}
\maketitle

\section*{51}
Let $f$ be the function passed as input to Simon's algorithm. When Simon's algorithm returns $a=0$ it is claiming that the function from $\left\{0,1\right\}^n \rightarrow \left\{0,1\right\}^n$ is a permutation. During the measurement of the first $n$ bits in Simon's algorithm, you get a uniformly distributed $y$ at random such that $y*a=0$. Since $a=0$ in this case, you get a uniformly distributed measurement of $y$. If $f$ is a permutation, then there is a uniform distribution that $y$ in the range of $f$ is chosen from input $x$, so Simon's algorithm correctly handles the case where $a=0$.

\section*{52}
\subsection*{a}
\[
\begin{bmatrix}
1 &  1 \\
1 & -1
\end{bmatrix}
\]
\subsection*{b}
\[
\begin{bmatrix}
1 &  1 \\
1 & -1
\end{bmatrix}
\]
\subsection*{c}
\[
\begin{bmatrix}
1 &  1 \\
1 & -1
\end{bmatrix}
\begin{bmatrix}
1 &  1 \\
1 & -1
\end{bmatrix}
\]
\subsection*{d}
\[
\begin{bmatrix}
1 &  1 &  1 &  1 \\
1 & -1 &  1 & -1 \\
1 &  1 & -1 & -1 \\
1 & -1 & -1 &  1 
\end{bmatrix}
\begin{bmatrix}
a \\
b \\
c \\
d \\
\end{bmatrix}
=
\begin{bmatrix}
a + b + c + d \\
a - b + c - d \\
a + b - c - d \\
a - b - c + d
\end{bmatrix}
\]
\subsection*{e}
\[
\begin{bmatrix}
1 &  1 \\
1 & -1
\end{bmatrix}
\begin{bmatrix}
\sqrt{a^2 + b^2} \\
\sqrt{c^2 + d^2}
\end{bmatrix}
=
\begin{bmatrix}
\sqrt{a^2 + b^2} + \sqrt{c^2 + d^2}\\
\sqrt{a^2 + b^2} - \sqrt{c^2 + d^2}
\end{bmatrix}
\]
\subsection*{f}
$\begin{bmatrix}
1 &  1 \\
1 & -1
\end{bmatrix}
\begin{bmatrix}
\sqrt{a^2 + b^2} \\
\sqrt{c^2 + d^2}
\end{bmatrix}
\begin{bmatrix}
1 &  1 \\
1 & -1
\end{bmatrix}
\begin{bmatrix}
\sqrt{a^2 + c^2} \\
\sqrt{b^2 + d^2}
\end{bmatrix}
=
\begin{bmatrix}
\sqrt{a^2 + b^2} + \sqrt{c^2 + d^2}\\
\sqrt{a^2 + b^2} - \sqrt{c^2 + d^2}
\end{bmatrix}
\begin{bmatrix}
\sqrt{a^2 + c^2} + \sqrt{b^2 + d^2} \\
\sqrt{a^2 + c^2} - \sqrt{b^2 + d^2}
\end{bmatrix}
=
$
\end{document}
