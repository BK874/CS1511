\documentclass[letterpaper,notitlepage,twoside]{article}

% Basic imports, increase margins...
\usepackage[margin=0.75in]{geometry}
\usepackage{amssymb}
\usepackage{amsmath}

% Finite State Machine stuff
\usepackage{pgf}
\usepackage{tikz}
\usetikzlibrary{arrows,automata}

% Format tables nicely
\usepackage[latin1]{inputenc}
\usepackage{array}
\usepackage{booktabs}
\setlength{\heavyrulewidth}{1.5pt}
\setlength{\abovetopsep}{4pt}

\usepackage{amsfonts} 
\usepackage{amssymb}
\usepackage{amsmath,amsthm}

\renewcommand{\implies}{\Rightarrow} % redefine command "implies"  
\renewcommand{\iff}{\Leftrightarrow} % double arrow
\newcommand{\maps}{\rightarrow} % define command "map" 
\newcommand{\union}{\cup}
\newcommand{\intersect}{\cap}
\newcommand{\N}{\mathbb{N}} % natural number 
\newcommand{\Q}{\mathbb{Q}} % rational number 
\newcommand{\R}{\mathbb{R}} % real number 
\newcommand{\Z}{\mathbb{Z}} % integers 
\newcommand\tab[1][1cm]{\hspace*{#1}} %\tab command

% Add more packages that you use here...

\begin{document}
\title{Homework 15}
\author{Joe Baker, Brett Schreiber, Brian Knotten}
\maketitle

\section*{23}

\section*{24}

\subsection*{a}
Problem 6.5: Show for every $k > 0$ that PH contains languages whose circuit complexity is $\Omega\left(n^k\right)$.
\\\\
Proof:
\\
Let $C$ be a circuit with complexity at of at least $n^k$. We know that such a circuit must exist by Theorem 6.22 from the book. We can construct a boolean formula $F$ from the gates of $C$ with $k$ quantifiers over the boolean formula. Now let $L$ be the language of all variable assignments for $k$ which satisfy $F$. Since $\left| C \right|$ is polynomial, $C$ can decide if an input is valid for $F$ in polynomial time. So there must exist a TM $M$ with $k$ advice tapes (from the quantifiers) that can decide the input with its advice tapes in polynomial time. Thus for each $k > 0$, there is a language in PH whose circuit complexity is $\Omega\left(n^k\right)$.

\subsection*{b}
Problem 6.6: Show for every $k > 0$ that $\Sigma_2^p$ contains languages whose circuit complexity is $\Omega\left(n^k\right)$.
\\\\
Proof:
\\


\subsection*{c}
Problem 6.: Show that if P = NP, then there is a language in EXP that requires circuits of size $\frac{2^n}{n}$.
\\\\
Proof:
\\


\end{document}
