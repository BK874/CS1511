\documentclass[letterpaper,notitlepage,twoside]{article}

% Basic imports, increase margins...
\usepackage[margin=0.75in]{geometry}
\usepackage{amssymb}
\usepackage{amsmath}

\usepackage{microtype}
\usepackage{listings}
\usepackage{framed}
\usepackage{wrapfig}

% Finite State Machine stuff
\usepackage{pgf}
\usepackage{tikz}
\usetikzlibrary{arrows,automata}

% Format tables nicely
\usepackage[latin1]{inputenc}
\usepackage{array}
\usepackage{booktabs}
\setlength{\heavyrulewidth}{1.5pt}
\setlength{\abovetopsep}{4pt}

\usepackage{amsfonts} 
\usepackage{amssymb}
\usepackage{amsmath,amsthm}

\renewcommand{\implies}{\Rightarrow} % redefine command "implies"  
\renewcommand{\iff}{\Leftrightarrow} % double arrow
\newcommand{\maps}{\rightarrow} % define command "map" 
\newcommand{\union}{\cup}
\newcommand{\intersect}{\cap}
\newcommand{\N}{\mathbb{N}} % natural number 
\newcommand{\Q}{\mathbb{Q}} % rational number 
\newcommand{\R}{\mathbb{R}} % real number 
\newcommand{\Z}{\mathbb{Z}} % integers 
\newcommand\tab[1][1cm]{\hspace*{#1}} %\tab command

% Add more packages that you use here...
\usepackage{braket}

\begin{document}
\title{Homework 33}
\author{Joe Baker, Brett Schreiber, Brian Knotten}
\maketitle

\section*{61}
Show that $NP = L-PCP(log(n))$.
\\\\
First we must show that $L-PCP(log(n)) \subseteq NP$. Let $l \in L-PCP(log(n))$, then $l$ can be decided by a probabilistically checked logspace machine $M$ using $r(n)$ random bits and one pass over the proof $\pi$. $M$ only has 1-way access to the proof $\pi$, but an $NP$ problem can be solved with turing machines with 2-way access to a certificate. From $M$ we can construct a Turing machine $M'$ which can accept in polynomial time using $\pi$ as the certificate and following the execution of $M$.
\\\\
Next we show that $NP \subseteq L-PCP(log(n))$.
\end{document}
