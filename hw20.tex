\documentclass[letterpaper,notitlepage,twoside]{article}

% Basic imports, increase margins...
\usepackage[margin=0.75in]{geometry}
\usepackage{amssymb}
\usepackage{amsmath}

% Finite State Machine stuff
\usepackage{pgf}
\usepackage{tikz}
\usetikzlibrary{arrows,automata}

% Format tables nicely
\usepackage[latin1]{inputenc}
\usepackage{array}
\usepackage{booktabs}
\setlength{\heavyrulewidth}{1.5pt}
\setlength{\abovetopsep}{4pt}

\usepackage{amsfonts} 
\usepackage{amssymb}
\usepackage{amsmath,amsthm}

\renewcommand{\implies}{\Rightarrow} % redefine command "implies"  
\renewcommand{\iff}{\Leftrightarrow} % double arrow
\newcommand{\maps}{\rightarrow} % define command "map" 
\newcommand{\union}{\cup}
\newcommand{\intersect}{\cap}
\newcommand{\N}{\mathbb{N}} % natural number 
\newcommand{\Q}{\mathbb{Q}} % rational number 
\newcommand{\R}{\mathbb{R}} % real number 
\newcommand{\Z}{\mathbb{Z}} % integers 
\newcommand\tab[1][1cm]{\hspace*{#1}} %\tab command

% Add more packages that you use here...

\begin{document}
\title{Homework 20}
\author{Joe Baker, Brett Schreiber, Brian Knotten}
\maketitle

\section*{32}
\subsection*{a}
First, it must be proven that $IP' \subseteq IP$.
Every time the verifier would ask the prover a question, instead, ask it $n$ times. $n$ can be so large, (say, 1000), that the
probability gets so high that one of the responses is guarunteed to be true. And thus we have a deterministic prover. \\\\

Next, it must be proven that $IP \subseteq IP'$.
Any $IP$ protocol is also $IP'$, because the definitions match in every way, except the prover's chance of success.
But a deterministic prover's chance of an accurate response $= 1 > 2/3$.

\subsection*{b}
Prove that $IP \subseteq PSPACE$.

Consider that an exponential number of proofs can be searched for in a proof tree using polynomial space by checking one branch at a time. Similarly, for any verifier $V$ in an $IP$ problem, a $PSPACE$ prover can be the prover that considers all possible sequences of messages that the verifier could send. The prover considers one branch of messages at a time using only a polynomial amount of space. So the prover is in $PSPACE$.

\subsection{c}
$IP' \subseteq IP$, because any language in $IP'$ trivially meets the same requirements for completeness and soundness for $IP$.

$IP \subseteq IP'$, because you can ask for enough advices such that one of the advices causes $V$ to accept $x$. Again, this uses the probablistic method as our other proofs.

\subsection{d}
To show that $NP \subseteq IP'$, have the prover send the advice that causes the verifier to accept. This works deterministically with probability of accepting correct inputs $1 > 2/3$, and probability of rejecting incorrect inputs $0$. So $NP \subseteq IP$. \\\\

To show that $IP' \subseteq NP$, ask the prover $n$ times. Again, $n$ can be so large that the probability gets so high that one of the responses is guarunteed to be true. And thus we have a deterministic prover. That is, one of the advices can be used in an $NP$ problem to decide $L$. \\\\

\section*{33}
It is trivial to prove that $IP' \subseteq IP$, since given that a prover $P$ in $IP'$ for all $x \in L$, then for each $x \in L$ there exists a prover, specifically, $P$. \\\\

Next it must be proven that $IP \subseteq IP'$. Consider a prover $P$ which is $\bigcup_{i = 0}^n P_i$ where $P_i$ is a prover for $x \in L$. All these $P_i$ are given from the definition of $IP$. 

\section*{34}
First, it must be proved that $BP \cdot NP \subseteq AM\left[2\right]$. Given a language $L \in BP \cdot NP$, let $L$ be the subject to an interactive proof. So the prover and verifier both have access to a given string $x$, and the prover $P$ has to prove to the verifier $V$ that $x \in L$. Since $L \ in BP \cdot NP$, there exists a certain set of advice which can cause a reducer $R$ to successfully reduce $x$ into an instance of 3SAT, $x'$. The prover has an unlimited amount of time to try different advices and successfully reduce $x$ into $x'$. Then, the prover can send over the reduction and the verifier can perform it. The prover can then decide the satisfiability of $x'$ and send over the satisfying assignments to the verifier.

\end{document}
