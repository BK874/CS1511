\documentclass[letterpaper,notitlepage,twoside]{article}

% Basic imports, increase margins...
\usepackage[margin=0.75in]{geometry}
\usepackage{amssymb}
\usepackage{amsmath}

% Finite State Machine stuff
\usepackage{pgf}
\usepackage{tikz}
\usetikzlibrary{arrows,automata}

% Format tables nicely
\usepackage[latin1]{inputenc}
\usepackage{array}
\usepackage{booktabs}
\setlength{\heavyrulewidth}{1.5pt}
\setlength{\abovetopsep}{4pt}

% Add more packages that you use here...
\usepackage{amsfonts} 
\usepackage{amssymb}
\usepackage{amsmath,amsthm}

\renewcommand{\implies}{\Rightarrow} % redefine command "implies"  
\renewcommand{\iff}{\Leftrightarrow} % double arrow
\newcommand{\maps}{\rightarrow} % define command "map" 
\newcommand{\union}{\cup}
\newcommand{\intersect}{\cap}
\newcommand{\N}{\mathbb{N}} % natural number 
\newcommand{\Q}{\mathbb{Q}} % rational number 
\newcommand{\R}{\mathbb{R}} % real number 
\newcommand{\Z}{\mathbb{Z}} % integers 
\newcommand\tab[1][1cm]{\hspace*{#1}} %\tab command

\begin{document}
\title{Homework 5}
\author{Joe Baker, Brett Schreiber, Brian Knotten}
\maketitle

\section*{7}

\subsection*{a}
A valid computation history of the given Turing Machine $M$ on the input $I = 101\textvisiblespace$ is: \\
$H = \#q_{0}101\textvisiblespace\#1q_{1}01\textvisiblespace\#11q_{1}1\textvisiblespace\#110q_{0}\textvisiblespace \#110q_{0}\textvisiblespace \#110q_{0}\textvisiblespace \#110q_{0}\textvisiblespace \#...$ \\
This computation history is infinite because $M$ does not halt on an input with an even number of $1$s. \\
\subsection*{b}
Define the macros:
\begin{itemize}
\item $PLACE(j)$ represents an arithmetic expression that returns the digit in location $j$ in $H$
\item $SAME(i, j, k, l)$ represents a logical expression that will be true iff digits $i$ through $j$ are identical to digits $k$ through $l$
\item $STATE(i)$ represents a local expression that will be true iff only digit $i$ of $H$ represents a state in $Q$
\item $TABLE(i, j)$ represents a local expression that will be true iff digits $i$, $i+1$ and $i+2$ in $H$ represents a tape symbol, state in $Q$ and a tape symbol respectively, and digits $j$, $j+1$ and $j+2$ in $H$ evolve properly from digits $i$, $i+1$, and $i+2$ according to $M$.
\item $DIGITS(i)$ represents the number of digits in a given integer $i$.
\item $FIRST(i, j)$ represents the $i$ most significant digits of the integer $j$.
\end{itemize}

Let $|\Gamma| = |\{q_0, q_1, q_h, \#, \textvisiblespace, 0, 1\}| = 7$ \\

Let $DIGITS(i) = \lceil log_{|\Gamma|}i \rceil$ \\ 
Let $FIRST(i, j) = \lfloor j \mathbin{/} |\Gamma|^{(DIGITS(j) - i)} \rfloor$ \\

% starts right
$FIRST(5, H) = \#q_0101$


% ends right
There exists some integer $j$ such that $PLACE(j_{end}) = q_h$ \\
and for all $k$ such that $0 < k < j, PLACE(k) \notin \{q_0, q_1, q_h, \# \}$ \\

$PLACE(0) = \#$

\end{document}