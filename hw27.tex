\documentclass[letterpaper,notitlepage,twoside]{article}

% Basic imports, increase margins...
\usepackage[margin=0.75in]{geometry}
\usepackage{amssymb}
\usepackage{amsmath}

% Finite State Machine stuff
\usepackage{pgf}
\usepackage{tikz}
\usetikzlibrary{arrows,automata}

% Format tables nicely
\usepackage[latin1]{inputenc}
\usepackage{array}
\usepackage{booktabs}
\setlength{\heavyrulewidth}{1.5pt}
\setlength{\abovetopsep}{4pt}

\usepackage{amsfonts} 
\usepackage{amssymb}
\usepackage{amsmath,amsthm}

\renewcommand{\implies}{\Rightarrow} % redefine command "implies"  
\renewcommand{\iff}{\Leftrightarrow} % double arrow
\newcommand{\maps}{\rightarrow} % define command "map" 
\newcommand{\union}{\cup}
\newcommand{\intersect}{\cap}
\newcommand{\N}{\mathbb{N}} % natural number 
\newcommand{\Q}{\mathbb{Q}} % rational number 
\newcommand{\R}{\mathbb{R}} % real number 
\newcommand{\Z}{\mathbb{Z}} % integers 
\newcommand\tab[1][1cm]{\hspace*{#1}} %\tab command

% Add more packages that you use here...
\usepackage{braket}

\begin{document}
\title{Homework 27}
\author{Joe Baker, Brett Schreiber, Brian Knotten}
\maketitle

\section*{49}
\subsection*{a}
$v = \frac{1}{\sqrt{4}} \ket{00} + \frac{1}{\sqrt{4}} \ket{01} + \frac{1}{\sqrt{4}} \ket{10} + \frac{1}{\sqrt{4}} \ket{11}$
Since the coefficients are the square-root of the probability, the probability of measuring any of $00, 01, 10, 11$ is $\frac{1}{4}$.
\subsection*{b}
Since $\ket{xy}$ can be rewritten as $\ket{x}\ket{y}$, we can simplify the problem into: \\
\begin{align*}
  &= \frac{1}{\sqrt{4}} \ket{00} + \frac{1}{\sqrt{4}} \ket{01} + \frac{1}{\sqrt{4}} \ket{10} + \frac{1}{\sqrt{4}} \ket{11} \\
  &= \frac{1}{\sqrt{4}} \ket{0_1} \ket{0_2} + \frac{1}{\sqrt{4}} \ket{0_1} \ket{1_2} + \frac{1}{\sqrt{4}} \ket{1_1} \ket{0_2} + \frac{1}{\sqrt{4}} \ket{1_1}\ket{1_2} \\
  &= \frac{1}{2} \ket{0_1} \ket{0_2} + \frac{1}{2} \ket{0_1} \ket{1_2} + \frac{1}{2} \ket{1_1} \ket{0_2} + \frac{1}{2} \ket{1_1}\ket{1_2} \\
  &= \frac{1}{\sqrt(2)}(\ket{0_1} + \ket{1_1})(\ket{0_2} + \ket{1_2})
\end{align*}
Measuring the first qubit would have a 1/2 probability of resulting in a 0 and a 1/2 probability of resulting in a 1, and the equation becomes either: \\
$v = 0_1 + \frac{1}{\sqrt{2}}(\ket{0_2} + \ket{1_2})$ \\
or: \\
$v = 1_1 + \frac{1}{\sqrt{2}}(\ket{0_2} + \ket{1_2})$ \\
both with probability 1/2.\\\\

And now the second qubit has a 1/2 chance of being either a 0 or 1 in both cases. So $v$ has a uniform 1/4 probability of being any of the following:\\
$v = 0_10_2$ \\
$v = 0_11_2$ \\
$v = 1_10_2$ \\
$v = 1_11_2$ \\

\subsection*{c}
The same proof as in b, but measure the second bit first. The probabilities will be the same due to the commutivity of multiplying the coefficient.\\
\section*{50}
\subsection*{a}

\subsection*{b}

\subsection*{c}

\end{document}
