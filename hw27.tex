\documentclass[letterpaper,notitlepage,twoside]{article}

% Basic imports, increase margins...
\usepackage[margin=0.75in]{geometry}
\usepackage{amssymb}
\usepackage{amsmath}

% Finite State Machine stuff
\usepackage{pgf}
\usepackage{tikz}
\usetikzlibrary{arrows,automata}

% Format tables nicely
\usepackage[latin1]{inputenc}
\usepackage{array}
\usepackage{booktabs}
\setlength{\heavyrulewidth}{1.5pt}
\setlength{\abovetopsep}{4pt}

\usepackage{amsfonts} 
\usepackage{amssymb}
\usepackage{amsmath,amsthm}

\renewcommand{\implies}{\Rightarrow} % redefine command "implies"  
\renewcommand{\iff}{\Leftrightarrow} % double arrow
\newcommand{\maps}{\rightarrow} % define command "map" 
\newcommand{\union}{\cup}
\newcommand{\intersect}{\cap}
\newcommand{\N}{\mathbb{N}} % natural number 
\newcommand{\Q}{\mathbb{Q}} % rational number 
\newcommand{\R}{\mathbb{R}} % real number 
\newcommand{\Z}{\mathbb{Z}} % integers 
\newcommand\tab[1][1cm]{\hspace*{#1}} %\tab command

% Add more packages that you use here...
\usepackage{braket}

\begin{document}
\title{Homework 27}
\author{Joe Baker, Brett Schreiber, Brian Knotten}
\maketitle

\section*{49}
\subsection*{a}
Let the state of $v = \alpha_{00}\ket{00} + \alpha_{01}\ket{01} + \alpha_{10}\ket{10} + \alpha_{11}\ket{11}$ where $Prob[$register's state is measured as $b_1b_2] = |\alpha_{b_1b_2}|^2$. \\

The coefficients can be rewritten to reflect the probabilities of each individual bit as $x_i$ and $y_i$ such that $Prob[b_1 = i] = x_i^2$ and $Prob[b_2 = i] = y_i^2$. \\

Using the multiplicative rule the $\alpha$ coefficients can be rewritten as follows: \\
\begin{align*}
Prob[b_1 = i \cap b_2 = j] &= Prob[b_1 = i] * Prob[b_2 = j] \\
|\alpha_{ij}|^2 &= x_i^2y_j^2 \\
\alpha_{ij} &= x_iy_j \\
\end{align*}

So the overall equation for $v = x_0y_0\ket{00} + x_0y_1\ket{01} + x_1y_0\ket{10} + x_1y_1\ket{11}$.

\subsection*{b}
The probability of measuring the second bit as a certain value given that the first bit measured to be a certain value can be described as follows:
\begin{align*}
Prob[b_2 = j | b_1 = i] &= \frac{Prob[b_2 = j \cap b_1 = i]}{Prob[b_1 = i} \\
&= \frac{x_i^2y_j^2}{x_i^2} \\
&= y_j^2 \\
\end{align*}
So the probability of measuring the second bit to be a certain value is independent of the first bit, so the equation stays as $v = x_0y_0\ket{00} + x_0y_1\ket{01} + x_1y_0\ket{10} + x_1y_1\ket{11}$.
\subsection*{c}
The probability of measuring the first bit as a certain value given that the second bit measured to be a certain value can be described as follows:
\begin{align*}
Prob[b_1 = i | b_2 = j] &= \frac{Prob[b_1 = i \cap b_2 = j]}{Prob[b_2 = j} \\
&= \frac{x_i^2y_j^2}{y_j^2} \\
&= x_i^2 \\
\end{align*}
So the probability of measuring the second bit to be a certain value is independent of the first bit, so the equation stays as $v = x_0y_0\ket{00} + x_0y_1\ket{01} + x_1y_0\ket{10} + x_1y_1\ket{11}$. Thus all three methods of measurement have the same probabilities.
\section*{50}
\subsection*{a}
If $x=y=1$ in the EPR experiment detailed in the book then, by symmetry, we can assume without loss of generality that Alice performs her rotation first. The initial state of the system is $\frac{1}{\sqrt{2}} \ket{00} + \frac{1}{\sqrt{2}} \ket{11}$ Applying the rotation to the first state, $\ket{00}$, results in $cos(\frac{\pi}{8}) \ket{00} + sin(\frac{\pi}{8}) \ket{10}$ and applying the rotation to the second state, $\ket{11}$, results in $cos(\frac{5\pi}{8}) \ket{01} + sin(\frac{5\pi}{8}) \ket{11}$. Therefore, Alice's rotation leaves the system in the state: $cos(\frac{\pi}{8}) \ket{00} + sin(\frac{\pi}{8}) \ket{10} + cos(\frac{5\pi}{8}) \ket{01} + sin(\frac{5pi}{8}) \ket{11}$. \\\\
Next, Bob performs his rotation; this results in $(cos(\frac{\pi}{8}) \ket{0} + sin(\frac{\pi}{8}) \ket{1})(cos(\frac{\pi}{8}) \ket{0}) - sin(\frac{\pi}{8})\ket{1})$ for the first states $\ket{00}$ and $\ket{10}$. The rotation on the other states $\ket{01}$ and $\ket{11}$ results in $(-sin(\frac{\pi}{8}) \ket{0} + cos(\frac{\pi}{8}) \ket{1})(sin(\frac{\pi}{8}) \ket{0} + cos(\frac{\pi}{8}) \ket{1})$
Therefore, Bob's rotation leaves the system in the state: \\
 $(cos(\frac{\pi}{8}) \ket{0} + sin(\frac{\pi}{8}) \ket{1})(cos(\frac{\pi}{8}) \ket{0}) - sin(\frac{\pi}{8})\ket{1}) + (-sin(\frac{\pi}{8}) \ket{0} + cos(\frac{\pi}{8}) \ket{1})(sin(\frac{\pi}{8}) \ket{0} + cos(\frac{\pi}{8}) \ket{1})$. \\\\
Note, however, that the resulting system state listed above is equivalent to the following: $(cos^{2}(\frac{\pi}{8}) - sin^{2}(\frac{\pi}{8})) \ket{00} - 2sin(\frac{\pi}{8})cos(\frac{\pi}{8}) \ket{01} + 2sin(\frac{\pi}{8})cos(\frac{\pi}{8}) \ket{10} + (cos^{2}(\frac{\pi}{8}) - sin^{2}(\frac{\pi}{8})) \ket{11}$.\\
Further note that: $cos^{2}(\frac{\pi}{8}) - sin^{2}(\frac{\pi}{8}) = cos(\frac{\pi}{4}) = 2sin(\frac{\pi}{8})cos(\frac{\pi}{8}) = sin(\frac{\pi}{4}) = \frac{1}{\sqrt{2}}$. \\
Thus, the final state of the system is: $\frac{1}{\sqrt{2}} \ket{00} - \frac{1}{\sqrt{2}} \ket{01} + \frac{1}{\sqrt{2}} \ket{10} + \frac{1}{\sqrt{2}} \ket{11}$. \\
Every coefficient in this state has the same absolute value, so when measured (regardless of the order of measurement) the system will yield each of the four values 00, 01, 10, and 11 with equals probability $\frac{1}{4}$. \\
Therefore the probability that $a=b$ and Alice and Bob win is $\frac{1}{2}$.
\subsection*{b}

\end{document}
