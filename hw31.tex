\documentclass[letterpaper,notitlepage,twoside]{article}

% Basic imports, increase margins...
\usepackage[margin=0.75in]{geometry}
\usepackage{amssymb}
\usepackage{amsmath}

\usepackage{microtype}
\usepackage{listings}
\usepackage{framed}

% Finite State Machine stuff
\usepackage{pgf}
\usepackage{tikz}
\usetikzlibrary{arrows,automata}

% Format tables nicely
\usepackage[latin1]{inputenc}
\usepackage{array}
\usepackage{booktabs}
\setlength{\heavyrulewidth}{1.5pt}
\setlength{\abovetopsep}{4pt}

\usepackage{amsfonts} 
\usepackage{amssymb}
\usepackage{amsmath,amsthm}

\renewcommand{\implies}{\Rightarrow} % redefine command "implies"  
\renewcommand{\iff}{\Leftrightarrow} % double arrow
\newcommand{\maps}{\rightarrow} % define command "map" 
\newcommand{\union}{\cup}
\newcommand{\intersect}{\cap}
\newcommand{\N}{\mathbb{N}} % natural number 
\newcommand{\Q}{\mathbb{Q}} % rational number 
\newcommand{\R}{\mathbb{R}} % real number 
\newcommand{\Z}{\mathbb{Z}} % integers 
\newcommand\tab[1][1cm]{\hspace*{#1}} %\tab command

% Add more packages that you use here...
\usepackage{braket}

\begin{document}
\title{Homework 31}
\author{Joe Baker, Brett Schreiber, Brian Knotten}
\maketitle

\section*{59}
First we will whow that the problem MAX-LINEAR of satisfying $n$ rational linear equations reduces from MAX-3SAT. We must convert the inputs of MAX-3SAT which is a Boolean formula of 3CNF form into the input of MAX-LINEAR which are rational linear equations. The conversion must be done such that the maximum number of satisfiable clauses in MAX-3SAT equals the maximum number of satisfiable clauses in MAX-LINEAR.
\\\\
The initial approach to this will be to apply the following steps:
\begin{enumerate}
\item For each clause in the formula, create a new equation
\item For each convert each variable $x$ from a Boolean value to an integer that can be 0 or 1
\item For each negated variable $\overline{x}$, write $(1-x)$
\item For each OR in the clause $(x \lor y)$, write $1 - (1 - x)(1 - y)$
\item For each clause, add an $= 1$ to assert that each equation is "true"
\end{enumerate}
Following these steps, a Boolean formula like:
\begin{gather*}
(x \vee y) \wedge \\
(x \vee \overline{y}) \wedge
\end{gather*}
would become the system of equations:
\begin{gather*}
1 - (1 - x)(1 - y)  = 1 \\
1 - (1 - x)y = 1
\end{gather*}
Initially, this approach can be done in poly-time and each corresponding equation will be true when the clause that derived it is true, but there are two problems. The first is that when multiple variables in the clause are true, the sum will be greater than one and the clause will be true but the equation will not. The second is that we will get an $xy$ term which is not linear.
\\\\
To solve these problems, we will make one more modification to the equations we've generated:
\begin{enumerate}
\item For each $xy$ term in an an equation, remove it and place it in a new equation as $1 - (1 - x) + (1 - y) = 1$
\end{enumerate}
There are now three cases to consider to show that the number of correct clauses corresponds to the number of correct equations:
\begin{enumerate}
\item If $x$ and $y$ are both true: Then the new second equation is true, but the original ($x + y$) is false
\item If one of $x$ and $y$ is true: Then the first equation is true, but the second equation is false.
\item Neither $x$ or $y$ is true: Then neither equation is true
\end{enumerate}
In all three cases, the number of correct clauses will match the number of correct equations. Finally note that this rule to transform variables can be extended to more than two variables. Now we have that MAX-3SAT $\leq$ MAX-LINEAR.
\\\\
Theorem 11.9 says that there is a constant $\rho < 1$ such that approximating MAX-3SAT is NP-Hard. Finally, since MAX-3SAT $\leq$ MAX-LINEAR, $\exists \rho < 1$ such that approximating MAX-LINEAR is NP-Hard.

\end{document}
