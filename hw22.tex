\documentclass[letterpaper,notitlepage,twoside]{article}

% Basic imports, increase margins...
\usepackage[margin=0.75in]{geometry}
\usepackage{amssymb}
\usepackage{amsmath}

% Finite State Machine stuff
\usepackage{pgf}
\usepackage{tikz}
\usetikzlibrary{arrows,automata}

% Format tables nicely
\usepackage[latin1]{inputenc}
\usepackage{array}
\usepackage{booktabs}
\setlength{\heavyrulewidth}{1.5pt}
\setlength{\abovetopsep}{4pt}

\usepackage{amsfonts} 
\usepackage{amssymb}
\usepackage{amsmath,amsthm}

\renewcommand{\implies}{\Rightarrow} % redefine command "implies"  
\renewcommand{\iff}{\Leftrightarrow} % double arrow
\newcommand{\maps}{\rightarrow} % define command "map" 
\newcommand{\union}{\cup}
\newcommand{\intersect}{\cap}
\newcommand{\N}{\mathbb{N}} % natural number 
\newcommand{\Q}{\mathbb{Q}} % rational number 
\newcommand{\R}{\mathbb{R}} % real number 
\newcommand{\Z}{\mathbb{Z}} % integers 
\newcommand\tab[1][1cm]{\hspace*{#1}} %\tab command

% Add more packages that you use here...

\begin{document}
\title{Homework 22}
\author{Joe Baker, Brett Schreiber, Brian Knotten}
\maketitle

\section*{36}

In an MAM protocol, Arthur makes the wrong decision on the first round with probablility $\frac{1}{3}$. But this probability can be as low as $\frac{1}{4^m}$, where $m$ is the number of bits that Merlin sends over. For proving some strings, Merlin may only need to send over 1 bit of information (like when proving an instance of GRAPH-NON-ISO). But Merlin can send $m$ bits instead as a proof against $m$ different tests. (In GRAPH-NON-ISO, Arthur sends over $m$ relabeled graphs, each randomly corresponding to either $G_0$ or $G_1$, to which Merlin responds with $m$ bits where the $i$th bit of $m$ corresponds to the $i$th graph that Arthur sent over). So Arthur has up to $m$ proofs to check against, giving him $m$ opportunities to reject, whereas before he may have only had just one. This lowers the probability of a wrong decision in GRAPH-NON-ISO from $\frac{1}{2}$ to $\frac{1}{2^m}$.


\section*{37}

\subsection*{a}

\subsubsection*{i}
Since the assignment of $x=1,y=0/1,z=1$ will satisfy this boolean formula, Merlin will send true answers for each function and integer. In the first step, Merlin will send the function $s(x)$ which is derived from an integer $S$ which is 1 iff the Boolean Formula $\exists x \forall y \exists z ( x \lor y \lor \overline{z}) \land (\overline{x} \lor \overline{y} \lor z)$ is true.
\begin{align*}
y \lor \overline{z} &\rightarrow (1 - y)(1 - (1 - z)) = 1 - z(1 - y)\\
x \lor y \lor \overline{z} &\rightarrow 1 - (1 - x)(1 - (1 - z(1 - y))) = 1 - (1 - x)(1 - y)z\\
\overline{y} \lor z &\rightarrow 1 - (1 - (1 - y))(1 - z) = 1 - y(1 - z)\\
\overline{x} \lor \overline{y} \lor z &\rightarrow 1 - (1 - (1 - x))(1 - (1 - y(1 - z))) = 1 - xy(1 - z)
\end{align*}
Using this, and changing $\exists \rightarrow \Sigma$, $\forall \rightarrow \Pi$, then $S$ is:
\begin{align*}
S(x)&=\Sigma_{x = 0}^1\Pi_{y = 0}^1\Sigma_{z = 0}^1 (1 - (1 - x)(1 - y)z)(1 - xy(1 - z)) = 1
\end{align*}
$S=1$ since Merlin knows the formula is satisfiable. Merlin will send
\begin{align*}
s(x)&=\Pi_{y = 0}^1\Sigma_{z = 0}^1 (1 - (1 - x)(1 - y)z)(1 - xy(1 - z))
\end{align*}
Additionally Merlin will return the integer 1 since he knows there is a true answer.

Alternative idea for 37.a.1:\\
Merlin will construct a function $s(x)$ as follows:

\begin{align*}
s(x) &= \Pi_{y = 0}^1\Sigma_{z = 0}^1 (x + y + (1 - z)) * ((1 - x) + (1 - y) + z) \\
     &= \Pi_{y = 0}^1\Sigma_{z = 0}^1 x(1 - x) + y(1 - x) + (1 - z)(1 - x) + x(1 - y) + y(1 - y) + (1 - z)(1 - y) + xz + yz + z(1 - z) \\
     &= \Pi_{y = 0}^1\Sigma_{z = 0}^1 x - x^2 + y - xy + 1 - z - x + xz + x - xy + y - y^2 + 1 - y - z + zy + xz + yz + z - z^2 \\
     &= \Pi_{y = 0}^1\Sigma_{z = 0}^1 x - x + x - x^2 + y - y + y - xy - xy - z - z + z + xz + xz - y^2 + 2zy - z^2 + 1 + 1 \\
     &= \Pi_{y = 0}^1\Sigma_{z = 0}^1 x - x^2 + y - 2xy - z + 2xz - y^2 + 2zy - z^2 + 2 \\
     &= \Pi_{y = 0}^1 (x - x^2 + y - 2xy - (0) + 2x(0) - y^2 + 2(0)y - (0)^2 + 2) + (x - x^2 + y - 2xy - (1) + 2x(1) - y^2 + 2(1)y - (1)^2 + 2) \\
     &= \Pi_{y = 0}^1 (x - x^2 + y - 2xy - y^2 + 2) + (x - x^2 + y - 2xy - 1 + 2x - y^2 + 2y - 1 + 2) \\
     &= \Pi_{y = 0}^1 3x - 2x^2 + 4y - 4xy - 2y^2 + 2 \\
     &= (3x - 2x^2 + 4(0) - 4x(0) - 2(0)^2 + 2) * (3x - 2x^2 + 4(1) - 4x(1) - 2(1)^2 + 2) \\
     &= (- 2x^2 + 3x + 2) * (- 2x^2 - x + 4) \\
     &= (- 2x^2 + 3x + 2) * (- 2x^2 - x + 4) \\
     &= 4 x^4 - 4 x^3 - 15 x^2 + 10 x + 8
\end{align*}

\begin{align*}
S &= s(0) + s(1) \\
  &= (4 (0)^4 - 4 (0)^3 - 15 (0)^2 + 10 (0) + 8) + (4 (1)^4 - 4 (1)^3 - 15 (1)^2 + 10 (1) + 8) \\
  &= 8 + 4 - 4 - 15  + 10  + 8 \\
  &= 11 \\
\end{align*}

\subsubsection*{ii}
\begin{align*}
s(x)&=\Sigma_{z = 0}^1 (1 - (1 - x)(1 - y)z)(1 - xy(1 - z)) \\
\end{align*}

\subsubsection*{iii}
Arthur will check that $s''(1/3) = s'(0) * s'(1)$.

\subsection*{b}
\subsubsection*{i}
Merlin will construct a linearized function $s(x)$ as follows:

\begin{align*}
s(x) &= \Pi_{y = 0}^1\Sigma_{z = 0}^1 (x + y + (1 - z)) * ((1 - x) + (1 - y) + z) \\
     &= \Pi_{y = 0}^1\Sigma_{z = 0}^1 x(1 - x) + y(1 - x) + (1 - z)(1 - x) + x(1 - y) + y(1 - y) + (1 - z)(1 - y) + xz + yz + z(1 - z) \\
     &= \Pi_{y = 0}^1\Sigma_{z = 0}^1 x - x + y - xy + 1 - z - x + xz + x - xy + y - y + 1 - y - z + yz + xz + yz + z - z \\
     &= \Pi_{y = 0}^1\Sigma_{z = 0}^1 2xz - 2xy + 2 - 2z + 2yz \\
     &= \Pi_{y = 0}^1 (2x(0) - 2xy + 2 - 2(0) + 2y(0)) + (2x(1) - 2xy + 2 - 2(1) + 2y(1)) \\
     &= \Pi_{y = 0}^1 - 2xy + 2 + 2x - 2xy + 2y \\
     &= \Pi_{y = 0}^1 - 4xy + 2 + 2x + 2y \\
     &= (- 4x(0) + 2 + 2x + 2(0)) * (- 4x(1) + 2 + 2x + 2(1)) \\
     &= (2 + 2x) * (- 2x + 4) \\
     &= -4 x + 4 x + 8 \\
     &= 8 \\
\end{align*}

\begin{align*}
S &= s(0) + s(1) \\
  &= 8 + 8 \\
  &= 16
\end{align*}

\subsubsection*{ii}
After receiving $r = 1 / 3$, Merlin will construct a linearized function $g(y)$ as follows:

\begin{align*}
g(y) &= \Sigma_{z = 0}^1 ((1/3) + y + (1 - z)) * ((1 - (1/3)) + (1 - y) + z) \\
     &= \Sigma_{z = 0}^1 -y + 2 y z + y/3 - z - z/3 + 20/9 \\
     &= \Sigma_{z = 0}^1 -2y/3 + 2 y z - 4z/3 + 20/9 \\
     &= (-2y/3 + 2 y (0) - 4(0)/3 + 20/9) + (-2y/3 + 2 y (1) - 4(1)/3 + 20/9) \\
     &= (-2y/3 + 20/9) + (-2y/3 + 2 y - 4/3 + 20/9) \\
     &= (2 y)/3 + 28/9
\end{align*}

\subsubsection*{iii}


\end{document}
