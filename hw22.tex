\documentclass[letterpaper,notitlepage,twoside]{article}

% Basic imports, increase margins...
\usepackage[margin=0.75in]{geometry}
\usepackage{amssymb}
\usepackage{amsmath}

% Finite State Machine stuff
\usepackage{pgf}
\usepackage{tikz}
\usetikzlibrary{arrows,automata}

% Format tables nicely
\usepackage[latin1]{inputenc}
\usepackage{array}
\usepackage{booktabs}
\setlength{\heavyrulewidth}{1.5pt}
\setlength{\abovetopsep}{4pt}

\usepackage{amsfonts} 
\usepackage{amssymb}
\usepackage{amsmath,amsthm}

\renewcommand{\implies}{\Rightarrow} % redefine command "implies"  
\renewcommand{\iff}{\Leftrightarrow} % double arrow
\newcommand{\maps}{\rightarrow} % define command "map" 
\newcommand{\union}{\cup}
\newcommand{\intersect}{\cap}
\newcommand{\N}{\mathbb{N}} % natural number 
\newcommand{\Q}{\mathbb{Q}} % rational number 
\newcommand{\R}{\mathbb{R}} % real number 
\newcommand{\Z}{\mathbb{Z}} % integers 
\newcommand\tab[1][1cm]{\hspace*{#1}} %\tab command

% Add more packages that you use here...

\begin{document}
\title{Homework 22}
\author{Joe Baker, Brett Schreiber, Brian Knotten}
\maketitle

\section*{36}

In an MAM protocol, Arthur makes the wrong decision on the first round with probablility $\frac{1}{3}$. But this probability can be as low as $\frac{1}{4^m}$, where $m$ is the number of bits that Merlin sends over. For proving some strings, Merlin may only need to send over 1 bit of information (like when proving an instance of GRAPH-NON-ISO). But Merlin can send $m$ bits instead as a proof against $m$ different tests. (In GRAPH-NON-ISO, Arthur sends over $m$ relabeled graphs, each randomly corresponding to either $G_0$ or $G_1$, to which Merlin responds with $m$ bits where the $i$th bit of $m$ corresponds to the $i$th graph that Arthur sent over). So Arthur has up to $m$ proofs to check against, giving him $m$ opportunities to reject, whereas before he may have only had just one. This lowers the probability of a wrong decision in GRAPH-NON-ISO from $\frac{1}{2}$ to $\frac{1}{2^m}$.


\section*{37}

\subsection*{a}

\subsubsection*{1}
Since the assignment of $x=1,y=0/1,z=1$ will satisfy this boolean formula, Merlin will send true answers for each function and integer. In the first step, Merlin will send the function $s(x)$ which is derived from an integer $S$ which is 1 iff the Boolean Formula $\exists x \forall y \exists z ( x \lor y \lor \overline{z}) \land (\overline{x} \lor \overline{y} \lor z)$ is true.
\begin{align*}
y \lor \overline{z} &\rightarrow (1 - y)(1 - (1 - z)) = 1 - z(1 - y)\\
x \lor y \lor \overline{z} &\rightarrow 1 - (1 - x)(1 - (1 - z(1 - y))) = 1 - (1 - x)(1 - y)z\\
\overline{y} \lor z &\rightarrow 1 - (1 - (1 - y))(1 - z) = 1 - y(1 - z)\\
\overline{x} \lor \overline{y} \lor z &\rightarrow 1 - (1 - (1 - x))(1 - (1 - y(1 - z))) = 1 - xy(1 - z)
\end{align*}
Using this, and changing $\exists \rightarrow \Sigma$, $\forall \rightarrow \Pi$, then $S$ is:
\begin{align*}
S(x)&=\Sigma_{x = 0}^1\Pi_{y = 0}^1\Sigma_{z = 0}^1 (1 - (1 - x)(1 - y)z)(1 - xy(1 - z)) = 1
\end{align*}
$S=1$ since Merlin knows the formula is satisfiable. Merlin will send
\begin{align*}
s(x)&=\Pi_{y = 0}^1\Sigma_{z = 0}^1 (1 - (1 - x)(1 - y)z)(1 - xy(1 - z))
\end{align*}
Additionally Merlin will return the integer 1 since he knows there is a true answer.

\subsubsection*{2}
\begin{align*}
s(x)&=\Sigma_{z = 0}^1 (1 - (1 - x)(1 - y)z)(1 - xy(1 - z)) \\
\end{align*}

\subsubsection*{3}
Arthur will check that $s''(1/3) = s'(0) * s'(1)$.

\subsection*{b}
\subsubsection*{1}
Merlin will construct a linearized function $s(x)$ as follows:

\begin{align*}
s(x) &= \Pi_{y = 0}^1\Sigma_{z = 0}^1 (1 - (1 - x)(1 - y)z)(1 - xy(1 - z)) \\
     &= \Pi_{y = 0}^1\Sigma_{z = 0}^1 1 - z(1 - x)(1 - y) - xy(1 - z) + xyz(1 - x)(1 - y)(1 - z) \\
     &= \Pi_{y = 0}^1\Sigma_{z = 0}^1 1 - (z - xz)(1 - y) - xy + xyz + (xyz - xyz)(1 - y)(1 - z) \\
     &= \Pi_{y = 0}^1\Sigma_{z = 0}^1 1 - (z - xz)(1 - y) - xy + xyz + (0)(1 - y)(1 - z) \\
     &= \Pi_{y = 0}^1\Sigma_{z = 0}^1 1 - (z - xz)(1 - y) - xy + xyz \\
     &= \Pi_{y = 0}^1\Sigma_{z = 0}^1 1 - (z - xz) - y(z - zx) - xy + xyz \\
     &= \Pi_{y = 0}^1\Sigma_{z = 0}^1 1 - z + xz - yz + xyz - xy + xyz \\
     &= \Pi_{y = 0}^1\Sigma_{z = 0}^1 1 - z + xz - yz - xy + 2xyz \\
\end{align*}

\subsubsection*{2}
\subsubsection*{3}


\end{document}
