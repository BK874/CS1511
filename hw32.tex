\documentclass[letterpaper,notitlepage,twoside]{article}

% Basic imports, increase margins...
\usepackage[margin=0.75in]{geometry}
\usepackage{amssymb}
\usepackage{amsmath}

\usepackage{microtype}
\usepackage{listings}
\usepackage{framed}
\usepackage{wrapfig}

% Finite State Machine stuff
\usepackage{pgf}
\usepackage{tikz}
\usetikzlibrary{arrows,automata}

% Format tables nicely
\usepackage[latin1]{inputenc}
\usepackage{array}
\usepackage{booktabs}
\setlength{\heavyrulewidth}{1.5pt}
\setlength{\abovetopsep}{4pt}

\usepackage{amsfonts} 
\usepackage{amssymb}
\usepackage{amsmath,amsthm}

\renewcommand{\implies}{\Rightarrow} % redefine command "implies"  
\renewcommand{\iff}{\Leftrightarrow} % double arrow
\newcommand{\maps}{\rightarrow} % define command "map" 
\newcommand{\union}{\cup}
\newcommand{\intersect}{\cap}
\newcommand{\N}{\mathbb{N}} % natural number 
\newcommand{\Q}{\mathbb{Q}} % rational number 
\newcommand{\R}{\mathbb{R}} % real number 
\newcommand{\Z}{\mathbb{Z}} % integers 
\newcommand\tab[1][1cm]{\hspace*{#1}} %\tab command

% Add more packages that you use here...
\usepackage{braket}

\begin{document}
\title{Homework 32}
\author{Joe Baker, Brett Schreiber, Brian Knotten}
\maketitle

\section*{60}
\subsection*{a}
The following assignment satisfies the equation: $u_1 = 1, u_2 = 0, u_3 = 1$.
\begin{align*}
(1)(1) + (1)(0) + (1)(1) = 1 + 0 + 1 &= 0 \\
(1)(1) + (0)(0) = 1 + 0 &= 1 \\
(1)(1) + (1)(0) = 1 + 0 &= 1 \\
(0)(0) + (0)(1) = 0 + 0 &= 0 \\
(0)(1) + (1)(1) + (1)(0) = 0 + 1 + 0 &= 1
\end{align*}

\subsection*{b}
See attached hw32.py file.

\subsection*{c}
All 520 bits (80 per row):
\\\\
01011010010110100101101001011010010110100101101001011010010110100101101010100101
\\\\
10100101101001011010010110100101101001011010010110100101010110100101101001011010
\\\\
01011010010110100101101001011010010110101010010110100101101001011010010110100101
\\\\
10100101101001011010010110100101101001011010010110100101101001011010010110100101
\\\\
10100101010110100101101001011010010110100101101001011010010110100101101010100101
\\\\
10100101101001011010010110100101101001011010010110100101010110100101101001011010
\\\\
0101101001011010010110100101101001011010

\subsection*{d}
The first 8 bits represent the Walsh-Hadamard encoding $f := WH(u=101)$, the next 512 bits represent the Walsh-Hadamard encoding of $g := WH(u \times u)$. When looking up bits in $f$, the first bit will be $000 \odot u$ and the last bit $111 \odot u$.
\\\\
When looking up the bits in $g$, consider $u \times u = \left( u_1u_1, u_1u_2, u_1u_3, u_2u_1, u_2u_2, u_2u_3, u_3u_1, u_3u_2, u_3u_3\right)$. To find the value of $u_1u_2 + u_2u_2 + u_3u_3$ consider the binary number formed when making bits $u_1u_2, u_2u_2, u_3u_3 = 1$ and the rest 0 in the tuple $u \times u$. This would give us the binary value $010010001_b = 145_d =: g_{index}$. Finally add $g_{index} + 8 + 1$ to account for the 8 bits in $f$ and the offset of 1 to put it in the range of $[1,520]$. Thus to find the value of $u_1u_2 + u_2u_2 + u_3u_3$, look at bit 154 (from the left, in range $[1,520]$) to find the value (in this case it's 0).

\end{document}
