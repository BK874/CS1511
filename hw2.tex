\documentclass[letterpaper,notitlepage,twoside]{article}

% Basic imports, increase margins...
\usepackage[margin=0.75in]{geometry}
\usepackage{amssymb}

% Finite State Machine stuff
\usepackage{pgf}
\usepackage{tikz}
\usetikzlibrary{arrows,automata}

% Format tables nicely
\usepackage[latin1]{inputenc}
\usepackage{array}
\usepackage{booktabs}
\setlength{\heavyrulewidth}{1.5pt}
\setlength{\abovetopsep}{4pt}

% Add more packages that you use here...
\newcommand{\block}[2] {
\vbox{1}
\vbox{2}
}

% Standard math font, symbols, and layout for theorems
\usepackage{amsfonts} 
\usepackage{amssymb}
\usepackage{amsmath,amsthm} 

\begin{document}
\title{Homework 2}
\author{Joe Baker, Brett Schreiber, Brian Knotten}
\maketitle


\section*{3}
% Description of problem
Post's Correspondence Problem can be proved to be undecidable by reducing $A_{TM}$ to $PCP$, where $A_{TM}$ is the acceptance problem for a Turing Machine. This is done by transforming the Turing Machine and input string for $A_{TM}$ into tiles as input for $PCP$ as described in the Wikipedia page, and returning the value of $PCP$. However, there are a few details to work out, such as:
\begin{enumerate}
\item How to guarantee that the solution must start with a given block?
\item Dealing with boundaries between states.
\item What happens if $PCP$ doesn't halt?
\item Can $PCP$ have different results for the same input from $A_{TM}$?
\item Can the first move of the input Turing Machine be left?
\item What if the input string to the TM is $\epsilon$?
\end{enumerate}
% First issue
Issue 1:\\
$PCP$ needs the initial block to be first in its match\\\\
Solution:\\
Certain inputs to $PCP$ can cause it to return a match with less tiles than what are necessary to create a valid TM computation history. To remedy this, create a new problem $MPCP$ which takes modified $PCP$ parameters as input and outputs whether a match can be found with the tiles from $PCP$. $MPCP$ will work exactly the same as $PCP$ except it will first modify each tile in the following ways:\\\\
\begin{itemize}
\item Transform the $q_0$ tile from $\frac{t_0}{b_0} \rightarrow \frac{*t_0}{*b_0*}$.
\item Transform the rest of the tiles $q_1...q_N$ from $\frac{t_i}{b_i} \rightarrow \frac{*t_i}{b_i*}$.
\item Add a new tile $q_{N+1} = \frac{*\square}{\square}$ to the input to $MPCP$ to consume the trailing new $*$ letter.
\end{itemize}
With this new configuration, $MPCP$ will need to match $q_0$ first since it is the only tile with $*$ to start, but any other tiles are not restricted further in matching. We now reduce $A_{TM}$ to $MPCP$ to make sure the first tile comes first, and then reduce $MPCP$ to $PCP$ to show that $PCP$ is undecidable.
\\\\
% Second issue
Issue 2:\\
Boundaries between states may not match up\\\\
Solution:\\
As described in the Wikipedia proof, the state strings are separated by a separator symbol (typically \#), and there are both "copy blocks" for each symbol $a \in \Gamma$, and a "transition block" for each position transition the machine can make. Because each of these blocks either has an equal number of symbols on the top and bottom or more symbols on the bottom, two states on the top and bottom will never match in a way that cannot be understood. Additionally, as defined in the Wikipedia proof, no transition blocks contain symbols other than $*$ to the left of a state symbol besides the $q_f$ blocks that are not used until the top reaches an accept state and the bottom needs to catch up. Therefore the only way states can line up in an incorrect way is if the blocks are not a match according to $PCP$.
\\\\
% Third issue
Issue 3:\\
What happens if $PCP$ doesn't halt on a transformed input from $A_{TM}$\\\\
Solution:\\
If $PCP$ doesn't halt on a transformed input from $A_{TM}$ then $A_{TM}$ won't halt and we will still have $PCP$ if and only if $A_{TM}$. 
\\\\
% Fourth issue
Issue 4:\\
Can $PCP$ have different results for the same input from $A_{TM}$?\\\\
Solution:\\
If $PCP$ matches on an input from $A_{TM}$ then it must match 1 or more copies of an acceptable computation history for $A_{TM}$. To show this, assume that $PCP$  matches partially up to tile $\frac{t_i}{b_i}$. If the input is held constant for each time $PCP$ is run, then $PCP$ will always choose the same tile after $\frac{t_i}{b_i}$ so the results cannot differ. However, $PCP$ can match the same correct computation history repeated more than one time. But in that case it will accept in any multiple of the computation history which is still the correct result and we still have $A_{TM}$ accepting if and only if $PCP$ accepts.
\\\\
% Fifth issue
Issue 5:\\
Can the first move of the input Turing Machine be left?\\\\
Solution:\\
If the input to $A_{TM}$ is $\left<M,w\right>$ where $M$ is a TM and $w$ is the input string, and the first move of $M$ is left past the end of the tape, the tiles for $PCP$ could not simulate the machine $M$. However, in this case, we could replace $M$ with an equivalent machine $M'$ which is able to prevent this behavior but still have equivalent results to $M$ with input $w$. So we can replace $M$ with $M'$ as input to $PCP$ for the reduction.
\\\\
% Sixth issue
Issue 6:\\
What if the input string to the TM is $\epsilon$?\\\\
Solution:\\
If the input to $A_{TM}$ is $\left<M,w\right>$ where $M$ is a TM and $w$ is the input string, and the $w$ is $\epsilon$, then replace the input string in the first tile with $\square$.

\end{document}