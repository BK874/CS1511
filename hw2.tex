\documentclass[letterpaper,notitlepage,twoside]{article}

% Basic imports, increase margins...
\usepackage[margin=0.75in]{geometry}
\usepackage{amssymb}

% Finite State Machine stuff
\usepackage{pgf}
\usepackage{tikz}
\usetikzlibrary{arrows,automata}

% Format tables nicely
\usepackage[latin1]{inputenc}
\usepackage{array}
\usepackage{booktabs}
\setlength{\heavyrulewidth}{1.5pt}
\setlength{\abovetopsep}{4pt}

% Add more packages that you use here...
\newcommand{\block}[2] {
\vbox{1}
\vbox{2}
}
\usepackage{amsfonts} % standard maths font 
\usepackage{amssymb} %standard maths symbol 
\usepackage{amsmath,amsthm} 

\begin{document}
\title{Homework 2}
\author{Joe Baker, Brett Schreiber, Brian Knotten}
\maketitle

% Example Dominoes: $\bigg[\dfrac{ab}{b}\bigg], \bigg[\dfrac{bc}{cac}\bigg], \bigg[\dfrac{ab}{cb}\bigg], \bigg[
% \dfrac{ca}{b}\bigg]$

\section*{3}
% Description of problem
Post's Correspondence Problem can be proved to be undecidable by reducing $A_{TM}$ to $PCP$, where $A_{TM}$ is the acceptance problem for a Turing Machine. This is done by transforming the Turing Machine and input string for $A_{TM}$ into tiles as input for $PCP$ as described in the Wikipedia page, and returning the value of $PCP$. However, there are a few details to work out, such as how to guarantee that the solution must start with a given block.\\\\
% First issue
Issue:\\
$PCP$ needs the initial block to be first in its match.\\\\
Solution:\\
Create a new problem $MPCP$ which takes modified $PCP$ parameters as input and outputs whether a match can be found with the tiles from $PCP$. $MPCP$ will work exactly the same as $PCP$ except it will first modify each tile in the following ways:\\\\
\begin{itemize}
\item Transform the $q_0$ tile from $\frac{t_1}{b_1} \rightarrow \frac{*t_1}{*b_1*}$.
\item Transform every other tile $q_i$ from $\frac{t_i}{b_i} \rightarrow \frac{*t_i}{b_i*}$.
\item Add a new tile $q_{N+1} = \frac{*\square}{\square}$ to the input to $MPCP$ to consume the trailing new $*$ letter.
\end{itemize}
With this new configuration, $MPCP$ will need to match $q_1$ first since it is the only tile with $*$ to start, but any other tiles are not restricted further in matching. We now reduce $A_{TM}$ to $MPCP$ to make sure the first tile comes first, and then reduce $MPCP$ to $PCP$ to show that $PCP$ is undecidable.
\\\\
% Second issue
% Third issue

% Notes
Thus 
$a$
The blocks of the PCP problem are created in such a way that any algorithm is forced to arrange the blocks in

Problem: Sometimes during the PCP algorithm there are steps which could be taken in more than one direction, causing ambiguity.

Problem: If the TM's first operation involves the left move, then a simulation of that TM should keep it in place. But a reduction to PCP does not account for this behavior.

Solution: Detect any first moves to the left in the reduction, and account for that by changing their transition into staying in place.

Problem: $w$ passed to the TM to solve $A_{TM}$ is $\epsilon$. This would cause the starting tile to be \# over \#$q_0$\#, and since there is no tile starting with $q_0$\#, the PCP algorithm will reject, even if the TM can operate on empty strings.

Solution: Check if $w = \epsilon$ in the reduction, and replace $w$ with $\square$.

Solution: Turn $\epsilon$

Lemma 1: The starting tile \#,\#$q_0$$w$ always appears first in the solution.
We can modify the strings on the top and bottom of each tile to make sure that $\#$,$\#$$q_0$$w$ must be the first tile. This can be achieved by turning $\alpha_1\alpha_2\alpha_3...\beta_1\beta_2\beta_3...$ into $*\alpha_1*\alpha_2*\alpha_3...\beta_1,*\beta_2*\beta_3*...$ for some unused character * and for each block in the problem not desired to be a starting block. The initial star guaruntees that the given block cannot go first, since the initial star in the top but not the bottom creates a discrepancy.

The intermediary stars allow the blocks to be connected together at any point. 

The final star on the bottom string allows the block to line up with another block which has an initial star.

For this reason, the starting block should also have a final star on the bottom, and stars in between each character.

A new block *@ over @ should be added for some unused character @ to allow the string to terminate.

Lemma 3: For a fixed $<M, s>$, the solution to PCP will always be the same.
Since TMs are deterministic, one configuration always implies the next configuration. And since $A_TM$ is reducible to PCP, then the solution to PCP will always be the same.


Lemma 3: If there exists a decider $R$ for PCP, then, since $HALT_{TM}$ is reducible to PCP, there are some inputs to PCP that will make $R$ loop forever. $A_{TM}$ halts iff PCP halts.

Lemma 4: TMs are deterministic, meaning that for a given TM $M$, and a given input $w$, $M$ will always run through the same configurations when processing $w$.

\end{document}