\documentclass[letterpaper,notitlepage,twoside]{article}
\usepackage[margin=0.75in]{geometry}
\usepackage{amssymb}
\usepackage{pgf}
\usepackage{tikz}
\usetikzlibrary{arrows,automata}
\usepackage[latin1]{inputenc}
\usepackage{array}
\usepackage{booktabs}
\setlength{\heavyrulewidth}{1.5pt}
\setlength{\abovetopsep}{4pt}

\begin{document}
\title{Homework 24}
\author{Joe Baker, Brett Schreiber, Brian Knotten}
\maketitle
\section*{42}

\section*{43}
Assume that $f^k(x)$ is not a one-way permutation of $x$. $f^k(x)$ is still a permutation, since $f(x)$ is a permutation. So therefore $f^k(x)$ is not one-way. That means an algorithm $A$ could figure out the input in polynomial time.

Construct an algorithm $B$ as follows:
Given $x$ and the one-way permutation $f$:
\tab Run $A$ on $f, x$ to get $y$ such that $f^k(y) = x$.
\tab Repeat the following procedure $k - 1$ times:
\tab\tab $y' := f(y)$
\tab\tab $y := y'$
\tab Return $y'$

Since $k$ is polynomial on $n$, then $B$ is a polynomial algorithm, since it loops only $k - 1$ times. $B$ returns the final value $y'$ such that $f(y') = x$. Therefore $B$ can reverse $f$. But $f$ is a one way permutation. It cannot be cracked in polynomial time. There is a contradiction. Therefore $f^k$ must also be a one-way permutation.

\end{document}
