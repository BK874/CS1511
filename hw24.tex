\documentclass[letterpaper,notitlepage,twoside]{article}
\usepackage[margin=0.75in]{geometry}
\usepackage{amssymb}
\usepackage{pgf}
\usepackage{tikz}
\usetikzlibrary{arrows,automata}
\usepackage[latin1]{inputenc}
\usepackage{array}
\usepackage{booktabs}
\setlength{\heavyrulewidth}{1.5pt}
\setlength{\abovetopsep}{4pt}

\begin{document}
\title{Homework 24}
\author{Joe Baker, Brett Schreiber, Brian Knotten}
\maketitle
\section*{42}

\section*{43}
Consider a one-way permutation $f$ on an input of size $n$ as a directed graph with $2^n$ nodes where each node corresponds to a binary string of size $n$. If $x$ and $y$ are nodes on the graph, and $f(x) = y$, then there is a directed edge from $x$ to $y$. Since $f$ is a function, there will only be one incoming and one outgoing edge from each node. If there were multiple outgoing edges from some node $x$, then that would imply there are two different results possible from $f(x)$, which can't be true since $x$ is a function. If there are multiple incoming edges for some node $x$ that would imply a loss of information, since a decryptor would not be able to tell which input caused the result to be $x$.

There are no cycles of size $n^k$ in the graph, because that would imply that a polynomial eavesdropper could, given $y$, find the $x$ such that $f(x) = y$. The eavesdropping algorithm would follow the cycle (that is, repeatedly apply the function) until it reaches $y$ again. Then it would know that the node it had just seen was $x$. So there can't be any cycles of polynomial size.

So it can be concluded that applying $f^k(x)$, that is, traversing a polynomial number of nodes, will result in a permutation that cannot be cracked.
\end{document}
