\documentclass[letterpaper,notitlepage,twoside]{article}

% Basic imports, increase margins...
\usepackage[margin=0.75in]{geometry}
\usepackage{amssymb}
\usepackage{amsmath}

% Finite State Machine stuff
\usepackage{pgf}
\usepackage{tikz}
\usetikzlibrary{arrows,automata}

% Format tables nicely
\usepackage[latin1]{inputenc}
\usepackage{array}
\usepackage{booktabs}
\setlength{\heavyrulewidth}{1.5pt}
\setlength{\abovetopsep}{4pt}

\usepackage{amsfonts} 
\usepackage{amssymb}
\usepackage{amsmath,amsthm}

\renewcommand{\implies}{\Rightarrow} % redefine command "implies"  
\renewcommand{\iff}{\Leftrightarrow} % double arrow
\newcommand{\maps}{\rightarrow} % define command "map" 
\newcommand{\union}{\cup}
\newcommand{\intersect}{\cap}
\newcommand{\N}{\mathbb{N}} % natural number 
\newcommand{\Q}{\mathbb{Q}} % rational number 
\newcommand{\R}{\mathbb{R}} % real number 
\newcommand{\Z}{\mathbb{Z}} % integers 
\newcommand\tab[1][1cm]{\hspace*{#1}} %\tab command

% Add more packages that you use here...

\begin{document}
\title{Homework 24}
\author{Joe Baker, Brett Schreiber, Brian Knotten}
\maketitle

\section*{42}
Consider the example where $|m| = 2$ and $|k| = 1$. Therefore there are $2^2 = 4$ possible messages and $2^1 = 2$ possible keys. An eavesdropper Carol intercepts an encrypted message $c$. The encryption-decryption scheme $(E, D)$ is public and so Carol knows it. But she doesn't know the key, so she tries all possible keys: $0$ and $1$. Carol runs $D_0(c)$ and gets $m_0$. Then she runs $D_1(c)$ and gets $m_1$. So Carol knows that the original message $m \in {m_0, m_1}$. So Carol can guess the message with probability $1/2$. However $|m| = 4$, meaning there are four possible messages: $m_0, m_1, m_2, m_3$. Therefore the probability of $E(m_2) = E(m_3) = 0$ and thus the distributions $E_{U_{n}}(m_0) \neq E_{U_{n}}(m_2)$. \\\\
More generally, let $(E, D)$ be a scheme with message size $m$ and key-size $n < m$. Let $m_{0} \in M$, the set of all possible messages, let $k_0 \in K$, the set of all possible keys, and let $c_{0} = E_{k_0}(m_0)$, the cipher text generated using $k_0$ when encrypting $m_0$. Then the probability of $c$ being generated using any arbitrary key $k \in K$ when encrypting $m_0$ is at least the probability of any randomly chosen key being $k_0$ i.e. $P(E_k(m_0) = c) \geq P(k = k_0) > 0$. \\\\
Now consider the set of all possible decryptions of $c_0$ $D=\{D_k(c_0) | k \in K\}$. Clearly, $D \subseteq M$. Note that because the decryption function is well-defined, there must be at least as many keys as possible decryptions, so $|D| \leq |K|$. Then by our assumption, $|D| \leq |K| < |M|$, implying that $\exists m_1 \in M$ such that $m_1 \notin D$. Therefore the probability of $c_0$ being generated using any key $k \in K$ when encrypting $m_1$ is 0 i.e. $P(E_k(m_1) = c_0) = 0$. Thus, there exists a pair of messages $m_0$, $m_1$ such that $E_{U_n}(m_0) \neq E_{U_n}(m_1)$ and $(E, D)$ is not a perfectly secret encryption scheme.


\section*{43}
Assume that $f^k(x)$ is not a one-way permutation of $x$. $f^k(x)$ is still a permutation, since $f(x)$ is a permutation. So therefore $f^k(x)$ is not one-way. That means an algorithm $A$, given $y$ and $f^k$ will output the $x$ such that $f^k(x) = y$ in polynomial time. \\\\
Construct an algorithm $B$ as follows: \\
Given $y, k$ and the one-way permutation $f$: \\
\tab Compute $z := f^{(k - 1)}(y)$ \\
\tab Compute $x := A(z, f^k)$ \\
\tab Return $x$\\\\
This algorithm $B$ computes $A(f^{(k - 1)}(y), f^k)$. Since $k$ is polynomial on $n$, then $B$ is a polynomial algorithm, since computes the function $f$ only $k - 1$ times. $B$ returns a value $x$ such that $f(x) = y$. Therefore $B$ can reverse $f$. But $f$ is a one way permutation. It cannot be cracked in polynomial time. There is a contradiction. Therefore $f^k$ must also be a one-way permutation.
\\\\
Alternative solution:
\\\\
Let $f$ be a one-way function and let $f^k(x) := f(f(...(f(x))))$ where $f$ is applied $k$ times. We will show that if $f$ is a one-way function, then $f^k$ is also a one-way function by contra-positive. Specifically we will show that if $f^k$ is not a one-way function, then $f$ is not a one-way function.
\\\\
Assume $f^k$ is not a one-way function. Then given $f^k(x) = y$, $x$ can be computed in polynomial time. We also know that $ f^k(x) = f(f^{k-1}(x')) = y$. Let $x := f^{k-1}(x')$. Since $f^k$ is not a one-way function, given $y$ we can determine $x$ in polynomial time. This means given $y$, we can compute $x$ such that $f(x) = y$ in polynomial time. Thus $f$ is also not a one-way function.

\end{document}
