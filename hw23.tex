\documentclass[letterpaper,notitlepage,twoside]{article}

% Basic imports, increase margins...
\usepackage[margin=0.75in]{geometry}
\usepackage{amssymb}
\usepackage{amsmath}

% Finite State Machine stuff
\usepackage{pgf}
\usepackage{tikz}
\usetikzlibrary{arrows,automata}

% Format tables nicely
\usepackage[latin1]{inputenc}
\usepackage{array}
\usepackage{booktabs}
\setlength{\heavyrulewidth}{1.5pt}
\setlength{\abovetopsep}{4pt}

\usepackage{amsfonts} 
\usepackage{amssymb}
\usepackage{amsmath,amsthm}

\renewcommand{\implies}{\Rightarrow} % redefine command "implies"  
\renewcommand{\iff}{\Leftrightarrow} % double arrow
\newcommand{\maps}{\rightarrow} % define command "map" 
\newcommand{\union}{\cup}
\newcommand{\intersect}{\cap}
\newcommand{\N}{\mathbb{N}} % natural number 
\newcommand{\Q}{\mathbb{Q}} % rational number 
\newcommand{\R}{\mathbb{R}} % real number 
\newcommand{\Z}{\mathbb{Z}} % integers 
\newcommand\tab[1][1cm]{\hspace*{#1}} %\tab command

% Add more packages that you use here...

\begin{document}
\title{Homework 23}
\author{Joe Baker, Brett Schreiber, Brian Knotten}
\maketitle

\section*{38 - Joe's Answer}
The paper explains how many developments were made very quickly while IP = PSPACE was discovered. The speed at which the discoveries happened was in large part due to email, because of how quickly new proofs could be shared. Babai explains at the end of the paper that the use of email to share thoughts and publish work has been just as problematic as it has been helpful. Aside from some mathematicians trying to steal each others work, he identifies a few issues that arise from email-proofs. The most interesting problems are who should be receiving each email, who should receive credit, and how should this work be split into published papers? The question of who should get what emails has no perfect answer since there are inherent trade-offs (sharing your work too soon to get it stolen, or too late and missing credit). However, to answer the other two questions, I'd like to imagine how the scenario may have played out if these mathematicians were all on the same campus instead of across the world. In this case, there would likely be fewer papers than there were emails, and with more names on each paper. This would model closer how each interaction through email helped to aid the next researcher in their discovery, and give each person credit where it was due, since their research may not have been the whole work but it was essential to getting there.

\section*{39}
\subsection*{a}
The interactive protocol to show that coNP $\subseteq$ IP has a problem when a boolean formula $\phi$ with variables $x_1,...,x_n$ is arithmetized and products are expanded. When $\phi$ is arithmetized to create function $f$, expanding the product will multiply the variables in such a way that the exponent of each variable can double in size, making the size of $f$ exponential. Since the verifier runs in polytime, it may not be able to read the function $f$.
\\\\
To solve this problem, assume that $\phi$ has the property that its variables $x_1,...,x_n$ occur in order in $\phi$ and that for each $x_i$ there is at most a single universal quantifier for variable $x_j$ appearing before the last occurrence of $x_i$. Now when $\phi$ is arithmetized, the resulting function $f$ will have a maximum degree of size $O(n)$.
\\\\
To see how the maximum degree is $O(n)$, consider how the $f$ will have the product for each variable expanded. At any point in the expansion we will have one of the two following scenarios:
\begin{enumerate}
\item $...\Pi_{x_i=0}^1f_{x_i}...\Pi_{x_j=0}^1f_{x_j}...$
\item $...\Pi_{x_{j'}=0}^1f_{x_{j'}}...\Pi_{x_i=0}^1f_{x_i}...\Pi_{x_j=0}^1f{x_j}...$
\end{enumerate}
In case 1, when each product is expanded from right-to-left we will have:
\begin{align*}
\left(...\Pi_{x_i=0}^1f_{x_i}...\right)f_{x_j=0}f_{x_j=1}
\end{align*}
and the part where $f_{x_j=0}$ will prevent the degrees of the variables from doubling for $x_j$. Additionally, any variables $x_k$ with $k > j$ will not be doubled in future expansions because they are not in future $f$ by definition (with the exception mentioned in case 2). By similar logic, variables $x_h$ with $h < i$ will not have been doubled in previous expansions.
\\\\
In case 2, we will have the same situation, but there will be one extra occurrence of the variable $x_j$, so the degree of $x_j$ will increase once from 1 to 2.
\\\\
Finally, each of the $n$ variables will have a max degree of $2$, so the degree of $f$ will be $O(n)$ and the verifier can validate the prover's response.

\subsection*{b}
Every size $n$ boolean formula $\phi$ can be transformed into a boolean formula $\psi$ with the sorted-variable property mentioned above and size $O(n^2)$.
\\\\
To transform $\phi$ into $\psi$ apply the following procedure from right to left until $\psi$ meets the sorted-variable property:
\begin{align*}
\text{Find the rightmost "fragment" of } \psi \text{ of the form: }& \forall x_i ... \forall x_j ... \forall x_{i'} f(x_i,...) \text{ where } i' > j > i \\
\text{Replace the fragment with the following: }& \forall x_i \exists y_i \text{ s.t. } (y_i=x_i) \land ... \forall x_j ... \forall x_{i'} f(y_i,...)
\end{align*}
Each of the $n$ variables can be out of place out of place (meaning there is a fragment) less than $n$ times, so there will be at most $O(n^2)$ fragments to fix. Since fixing each fragment adds 1 new variable (without adding a new fragment, because a fixed fragment adds an $\exists$ and a variable in correct order) , the resulting formula $\psi$ will have a maximum size of $O(n^2)$. 

\section*{40}
Alice and Bob both know a secret key $k$, a one time pad, which is the same bitlength as the message $m$.
Carol does not know any information about this secret key, and her best course of action for determining a random bit of the key is to guess.
So Carol has probability $1/2$ of guessing any bit of the key using a randomized polynomial algorithm $A$. \\\\
The xor operation has the property such that $m \oplus k = c$, where $c$ is the ciphertext, and $c \oplus k = m$. Moreover, the xor function is one-to-one, meaning that no other $k$ can derive $c$ from $m$ and vice versa. \\\\
Alice can encrypt her message $m$ using a one time pad $k$ to get $c$ using xor. Bob can similarly decrypt $c$ into $m$ using xor.
Since no other $k$ can derive $m$, and since Carol cannot determine any bit of $k$ with probability greater than $1/2$, it follows that Carol cannot derive any bit of $m$ with the same probability greater than $1/2$.
So Alice and Bob have computational security on $m$.

\section*{41}
Assume $P=NP$, let $f:\{0,1\}^{*} \rightarrow \{0,1\}^{*}$ be a one-way function, and let $y$ be an output of $f$. \\
Let $A$ be a non-deterministic poly-time Turing Machine that tries every possible $x$ until $f(x) = y$ and then returns $x$ i.e. $A$ solves the problem of inverting $f$. \\ \\
Because $A$ solves the problem of inverting a one-way function in poly-time, the problem is in $NP$.
By our assumption $P=NP$, so there the problem is also in $P$ and exists an efficient algorithm for inverting one-way functions.
Therefore one-way functions do not exist.

\end{document}
