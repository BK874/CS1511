\documentclass[letterpaper,notitlepage,twoside]{article}

% Basic imports, increase margins...
\usepackage[margin=0.75in]{geometry}
\usepackage{amssymb}
\usepackage{amsmath}

% Finite State Machine stuff
\usepackage{pgf}
\usepackage{tikz}
\usetikzlibrary{arrows,automata}

% Format tables nicely
\usepackage[latin1]{inputenc}
\usepackage{array}
\usepackage{booktabs}
\setlength{\heavyrulewidth}{1.5pt}
\setlength{\abovetopsep}{4pt}

\usepackage{amsfonts} 
\usepackage{amssymb}
\usepackage{amsmath,amsthm}

\renewcommand{\implies}{\Rightarrow} % redefine command "implies"  
\renewcommand{\iff}{\Leftrightarrow} % double arrow
\newcommand{\maps}{\rightarrow} % define command "map" 
\newcommand{\union}{\cup}
\newcommand{\intersect}{\cap}
\newcommand{\N}{\mathbb{N}} % natural number 
\newcommand{\Q}{\mathbb{Q}} % rational number 
\newcommand{\R}{\mathbb{R}} % real number 
\newcommand{\Z}{\mathbb{Z}} % integers 
\newcommand\tab[1][1cm]{\hspace*{#1}} %\tab command

% Add more packages that you use here...

\begin{document}
\title{Homework 23}
\author{Joe Baker, Brett Schreiber, Brian Knotten}
\maketitle

\section*{39}

\section*{40}
Alice and Bob both know a secret key $k$, a one time pad, which is the same bitlength as the message $m$.
Carol does not know any information about this secret key, and her best course of action for determining a random bit of the key is to guess.
So Carol has probability $1/2$ of guessing any bit of the key using a randomized polynomial algorithm $A$. \\\\
The xor operation has the property such that $m \oplus k = c$, where $c$ is the ciphertext, and $c \oplus k = m$. Moreover, the xor function is one-to-one, meaning that no other $k$ can derive $c$ from $m$ and vice versa. \\\\
Alice can encrypt her message $m$ using a one time pad $k$ to get $c$ using xor. Bob can similarly decrypt $c$ into $m$ using xor.
Since no other $k$ can derive $m$, and since Carol cannot determine any bit of $k$ with probability greater than $1/2$, it follows that Carol cannot derive any bit of $m$ with the same probability greater than $1/2$.
So Alice and Bob have computational security on $m$.

\section*{41}
Assume $P=NP$, let $f:\{0,1\}^{*} \rightarrow \{0,1\}^{*}$ be a one-way function, and let $y$ be an output of $f$. \\
Let $A$ be a non-deterministic poly-time Turing Machine that "guesses" every possible $x$ such that $f(x) = y$ i.e. 
$A$ solves the problem of inverting $f$. \\ \\
Because $A$ solves the problem of inverting a one-way function in poly-time, the problem is in $NP$.
By our assumption $P=NP$, so there the problem is also in $P$ and exists an efficient algorithm for inverting one-way functions.
Therefore one-way functions do not exist.

\end{document}
