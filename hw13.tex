\documentclass[letterpaper,notitlepage,twoside]{article}

% Basic imports, increase margins...
\usepackage[margin=0.75in]{geometry}
\usepackage{amssymb}
\usepackage{amsmath}
\usepackage{commath}

% Finite State Machine stuff
\usepackage{pgf}
\usepackage{tikz}
\usetikzlibrary{arrows,automata}

% Format tables nicely
\usepackage[latin1]{inputenc}
\usepackage{array}
\usepackage{booktabs}
\setlength{\heavyrulewidth}{1.5pt}
\setlength{\abovetopsep}{4pt}

\usepackage{amsfonts} 
\usepackage{amssymb}
\usepackage{amsmath,amsthm}

\renewcommand{\implies}{\Rightarrow} % redefine command "implies"  
\renewcommand{\iff}{\Leftrightarrow} % double arrow
\newcommand{\maps}{\rightarrow} % define command "map" 
\newcommand{\union}{\cup}
\newcommand{\intersect}{\cap}
\newcommand{\N}{\mathbb{N}} % natural number 
\newcommand{\Q}{\mathbb{Q}} % rational number 
\newcommand{\R}{\mathbb{R}} % real number 
\newcommand{\Z}{\mathbb{Z}} % integers 
\newcommand\tab[1][1cm]{\hspace*{#1}} %\tab command

% Add more packages that you use here...

\begin{document}
\title{Homework 13}
\author{Joe Baker, Brett Schreiber, Brian Knotten}
\maketitle

\section*{19}

\subsection*{a}
Problem 5.9 part (a): Show that EXACT INDSET $\in \pi_2^p$.
\\\\
Proof:\\
EXACT INDSET can be expressed as the following language in first-order logic:\\\\
EXACT INDSET = $\left\{ \left( G, k \right) \mid \forall \text{ Independent Sets } S_1 \in G,  \exists \text{ an Independent Set } S_2 \in G \text{ such that } \abs{S_2} = k \text{ and } \abs{S_2} >= \abs{S_1} \right\}$.\\\\
By the definition of $\pi_2^p$, EXACT INDSET $\in \pi_2^p$.

\subsection*{b}
Problem 5.11: Show that SUCCINCT SET-COVER $\in \Sigma_2^p$.
\\\\
Proof:\\
SUCCINCT SET COVER can be expressed as the following language in first-order logic:\\\\
SUCCINCT SET COVER = $\left\{ \left( S, k \right) \mid \exists \text{ } S' \subseteq S, \forall \text{ possible assignments of } S', \text{ then } \abs{S'} <= k \text{ and } S' \text{ true} \right\}$.\\\\
By the definition of $\Sigma_2^p$, SUCCINCT SET COVER $\in \Sigma_2^p$.

\subsection*{c}
Problem 5.13: Show that VC-DIMENSION $\in \Sigma_3^p$.
\\\\\
Proof:\\
VC-DIMENSION can be expressed as the following language in first-order logic:\\\\
VC-DIMENSION = $\left\{ \left( C, k \right) \mid \exists \text{ } X \subseteq U, \forall \text{ } X' \subseteq X, \exists \text{ } i, \text{ such that } \abs{X} >= k \text{ and } C_i \cup X = X' \right\}$.\\\\
By the definition of $\Sigma_3^p$, VC-DIMENSION $\in \Sigma_3^p$.

\subsection*{d}
Problem 5.9 part(b): Show that EXACT INDSET $\in $ DP.
\\\\
Proof:\\
Let $L_1 = \left\{ \left( G, k \right) \mid \exists \text{ an Independent Set } S, \text{ such that } \abs{S} = k \right\}$.\\
$L_1 \in \text{NP}$ because $L_1 \in \Sigma_1^p$ by definition and $\Sigma_1^p = \text{NP}$.
\\\\
Let $L_2 = \left\{ \left( G, k \right) \mid \forall \text{ Independent Sets } S, \text{ such that } \abs{S} <= k \right\}$.\\
$L_2 \in \text{co-NP}$ because it is in $\pi_1^p$ by definition and $\pi_1^p = \text{co-NP}$.
\\\\
Since the intersection of two sets requires both constraints to be met:\\
Let $L = L_1 \cap L_2 = \left\{ \left( G, k \right) \mid \forall \text{ Independent Sets } S_1, \exists \text{ an Independent Set } S_2, \text{ such that } \abs{S_1} <= k \text{ and } \abs{S_2} = k \right\}$ = EXACT IND SET.\\\\
Finally, since $L = L_1 \cap L_2$ and $L_1 \in \text{NP}$ and $L_2 \in \text{co-NP}$, then $L \in \text{DP}$.

\section*{20}
Problem 5.3: Show that if 3SAT is polynomial-time reducible to $\overline{\text{3SAT}}$, then PH = NP.
\\\\
Proof:\\
First we will show that if 3SAT is polynomial-time reducible to $\overline{\text{3SAT}}$, then NP = co-NP:\\
Since 3SAT $\le_P \overline{\text{3SAT}}$, then there is a bijective function $f: \left\{ 0,1 \right\}^* \rightarrow \left\{ 0,1 \right\}^*$ where $x \in $ 3SAT if and only if $f(x) \in \overline{\text{3SAT}}$. Since 3SAT is NP-complete, for each language $L_{\text{NP}} \in$ NP, there exists some bijective function $f_{L_{\text{NP}}}$ which reduces $L_{\text{NP}}$ to 3SAT. Similarly, since $\overline{\text{3SAT}}$ is co-NP-complete, for each language $L_{\text{co-NP}} \in$ co-NP, there exists some bijective function $f_{L_{\text{co-NP}}}$ which reduces $L_{\text{co-NP}}$ to $\overline{\text{3SAT}}$. Thus any language $L_{\text{NP}} \in $ NP can be reduced to some language $L_{\text{co-NP}} \in$ co-NP by applying $f_{\text{NP}}(f(f_{\text{co-NP}}(x)))$ for each $x \in L_{\text{NP}}$ and any language $L_{\text{co-NP}} \in $ co-NP can be reduced to some language $L_{\text{NP}} \in$ NP by applying $f^{-1}_{\text{co-NP}}(f^{-1}(f^{-1}_{\text{NP}}(x)))$ for each $x \in L_{\text{co-NP}}$. Thus 3SAT $\le_P \overline{\text{3SAT}}$ implies that any language in NP is also in co-NP, and any language in co-NP is also in NP. 
%Let $L \in \text{co-NP}$. Then $\overline{\text{3SAT}} \le_P L$ since $\overline{\text{3SAT}}$ is co-NP-complete. Next, 3SAT $\le_P \overline{\text{3SAT}} \le_P L$ by our original assumption, so $L$ is reducible to 3SAT. Since 3SAT $\le_P L$, there exists a TM with input and advice tapes which accepts $L$ in polynomial time, so $L \in$ NP. Thus, $L \in$ co-NP $\rightarrow L \in$ NP.\\
%Now let $L \in \text{NP}$. Then.... prove pt 1.b
\\\\
Next we will show that if NP = co-NP, then $\Sigma_2^p = \text{NP} = \pi_2^p = \text{co-NP}$:\\
Consider an arbitrary language $L \in$ NP. If NP = co-NP, then $L \in$ co-NP.\\
$L$ being in NP implies that for some string $x \in L$, there exists a certificate $c$ of polynomial size which a poly-time Turing machine can use to verify that $x \in L$. \\
$L$ being in co-NP implies that for some string $x \in L$, for all arbitrary "counter examples", a poly-time Turing machine can verify these counter examples do not forbid $x$ from being a member of $L$. \\

Consider an arbitrary language $M$ such that $M \in \Sigma_2^p$. This means that $M$ can be described as an existential statement followed by a universal. Since NP = co-NP, this latter universal statement can be rewritten as an existential statement, because instead of providing all counter examples, a verifier TM can be given a verifying certificate $c$. So $M$ can be rewrwitten as two existential statements, which can further be rewritten as a single existential statement requiring a pair of values which would both verify an input string $x \in M$. Since $M$ can be verified by a poly-time verifier and a certificate (containing a tuple), $M \in$ NP. Therefore, for all $M \in \Sigma_2^p \implies M \in$ NP. \\

It follows that any language $N \in \Pi_2^p \implies N \in$ co-NP, because $N$'s universal and existential statement can be rewritten as a universal statement requiring a pair of values. And since $N \in$ co-NP, $N \in$ NP. \\

This strategy can be applied to all levels of the polynomial hierarchy. Any given class $\Sigma_n^p$ can have its final existential or universal quantifier be rewritten as a universal or existential quantifier respectively and combined with the previous quantifier to expect a pair of values, which transforms the class into $\Sigma_{n-1}^p$, which itself can have its final  quantifier be rewritten and be combined with the previous quantifier to expect a 3-tuple of values. This can be repeatedly applied until the class $\Sigma_1^p$ is reached, which is NP. The same is true for any given class $\Pi_n^p$. \\

Finally, we can repeat the logic used to show that $\Sigma_2^p$ and $\pi_2^p$ collapse into NP and co-NP to show that higher levels of the polynomial time hierarchy collapse into NP and co-NP. Thus if 3SAT is polynomial-time reducible to $\overline{\text{3SAT}}$, then PH = NP.

\end{document}
