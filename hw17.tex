\documentclass[letterpaper,notitlepage,twoside]{article}

% Basic imports, increase margins...
\usepackage[margin=0.75in]{geometry}
\usepackage{amssymb}
\usepackage{amsmath}

% Finite State Machine stuff
\usepackage{pgf}
\usepackage{tikz}
\usetikzlibrary{arrows,automata}

% Format tables nicely
\usepackage[latin1]{inputenc}
\usepackage{array}
\usepackage{booktabs}
\setlength{\heavyrulewidth}{1.5pt}
\setlength{\abovetopsep}{4pt}

\usepackage{amsfonts} 
\usepackage{amssymb}
\usepackage{amsmath,amsthm}

\renewcommand{\implies}{\Rightarrow} % redefine command "implies"  
\renewcommand{\iff}{\Leftrightarrow} % double arrow
\newcommand{\maps}{\rightarrow} % define command "map" 
\newcommand{\union}{\cup}
\newcommand{\intersect}{\cap}
\newcommand{\N}{\mathbb{N}} % natural number 
\newcommand{\Q}{\mathbb{Q}} % rational number 
\newcommand{\R}{\mathbb{R}} % real number 
\newcommand{\Z}{\mathbb{Z}} % integers 
\newcommand\tab[1][1cm]{\hspace*{#1}} %\tab command

% Add more packages that you use here...

\begin{document}
\title{Homework 17}
\author{Joe Baker, Brett Schreiber, Brian Knotten}
\maketitle

\section*{27}
Assume that our coin-flip Turing machine $M$ doesn't require that $\rho$ be efficiently computable.
\\\\
Let $\rho = \frac{1}{\pi}$.
\\\\
The complete binary representation of $\frac{1}{\pi}$ is infinitely long. Any finite representation is not completely accurate, since $\frac{1}{\pi}$ is irrational.
\\\\
By the law of large numbers, since $\frac{1}{\pi}$ represented in binary is infinitely long with no repeated patterns,the probability that it must contain a substring of bits which is the encoding of a Turing machine which can solve the halting problem goes to 1 as the number of bits in $\frac{1}{\pi}$ goes to infinity.
\\\\
Since $M$ generates a Turing machine to decide the halting problem as an intermediate step, removing the requirement that $\rho$ be efficiently computable implies that $M$ can decide an undecidable language in polynomial time. 

\section*{28}
Show that $\left(\text{NP}\cup\text{co-NP}\right)\subseteq\text{PP}$.
\\\\
Proof:
\\
We will show that $\text{Boolean Satisfiability}\in\text{PP}$. Since Boolean Satisfiability is NP-complete, then any $L \in \text{NP} \rightarrow L \in \text{PP}$.
\\
Let $M$ be a probabilistic Turing machine which takes a Boolean formula $F$ as input. Then does the following:
\begin{enumerate}
\item Pick a random assignment of variables for $F$
\item Verify if $F$ is satisfied
\item If $F$ is satisfied, then accept
\item If $F$ is not satisfied, accept with $P(1/2)$ or reject with $P(1/2)$
\end{enumerate}
First note that each step can clearly be done in polynomial time. $P(\text{M accepts}) = 1/2 + P(\text{random assignment is satisfied})$ and since $P(\text{random assignment is satisfied}) > 0$, then $P(\text{M accepts}) > 1/2$ and $P(\text{M rejects}) < 1/2$. Thus Boolean Satisfiability $\in \text{PP}$ since $M$ is a Turing machine which meets the requirements for PP. So NP is contained in PP.
\\\\
Since PP is closed under complement and contains NP, it must also contain co-NP. Finally, $\left(\text{NP}\cup\text{co-NP}\right)\subseteq\text{PP}$.

\end{document}
