\documentclass[letterpaper,notitlepage,twoside]{article}

% Basic imports, increase margins...
\usepackage[margin=0.75in]{geometry}
\usepackage{amssymb}
\usepackage{amsmath}

% Finite State Machine stuff
\usepackage{pgf}
\usepackage{tikz}
\usetikzlibrary{arrows,automata}

% Format tables nicely
\usepackage[latin1]{inputenc}
\usepackage{array}
\usepackage{booktabs}
\setlength{\heavyrulewidth}{1.5pt}
\setlength{\abovetopsep}{4pt}

\usepackage{amsfonts} 
\usepackage{amssymb}
\usepackage{amsmath,amsthm}

\renewcommand{\implies}{\Rightarrow} % redefine command "implies"  
\renewcommand{\iff}{\Leftrightarrow} % double arrow
\newcommand{\maps}{\rightarrow} % define command "map" 
\newcommand{\union}{\cup}
\newcommand{\intersect}{\cap}
\newcommand{\N}{\mathbb{N}} % natural number 
\newcommand{\Q}{\mathbb{Q}} % rational number 
\newcommand{\R}{\mathbb{R}} % real number 
\newcommand{\Z}{\mathbb{Z}} % integers 
\newcommand\tab[1][1cm]{\hspace*{#1}} %\tab command

% Add more packages that you use here...

\begin{document}
\title{Homework 25}
\author{Joe Baker, Brett Schreiber, Brian Knotten}
\maketitle
\section*{44}
\subsection*{a}
Given a one-way function $f$, it is possible to create a new one-way function $g$ which runs in $O(n^2)$ time as follows: \\
On input $x$ of size $n$: \\
\tab Split the input $x$ into $log(n)$ chunks: $x_1, x_2...x_{log(n)}$. \\
\tab for each $x_i$: \\
\tab \tab Compute $f(x_i)$, keeping track of the number of steps $f$ takes. After $n^2$ steps, just output $0$.\\
\tab Return $f(x_1) || f(x_2) ... || f(x_{log(n)}$ where $||$ is the concatenation of the bitstrings. \\
\tab Some of these substrings will be $0$.\\\\
First, $g$ runs in $O(n^2)$ time, because $f$ performing $log(n)$ computations. So $g$ is the complexity of $f$ multiplied by $log(n)$. Since we stop $f$ after $n^2$ steps, the total runtime is $n^2 * log(n) = O(n^2)$. \\\\
Second, $g$ is a one way function, since $g = f_U$, and $f_U$ is one-way as proved below.

\subsection*{b}
$f$ is one way $\implies f_U$ is one-way. This can be proved by contrapositive, that $f_U$ is not one-way $\implies f$ is not one way. \\
Assume $f_U$ is not one-way. Then there exists an algorithm $A_U$ which given $y$ can produce the $x$ such that $f_U(x') = y'$ in polynomial time. Then you can construct an algorithm $A$ which given $y$ can produce the $x$ such that $f(x) = y$ in polynomial time.
\\\\
$A =$ on input $y$:
\begin{enumerate}
\item Generate $r = $ some number of random bits.
\item Construct the string $y' := y || r$.
\item Run $A$ on $y'$ to get $x'$.
\item If $f(x') = y$, return $x$, else, go back to step $1$.
\end{enumerate}

Since $A_U$ exists, then $f_U$ can be broken with greater than $\epsilon$ probability. Since there is a $1/2^{log(n)} = 1/n$ chance that $y'$ is picked correctly, those two probabilities multiplied together will mean that the probability of breaking $f$ is also greater than $\epsilon$ so, $f$ being a OWF $\implies f_U$ is a OWF.

\section*{45}
\subsection*{a}
Let $(E, D)$ be a semantically secure encryption scheme and let $f(x)$ be a function that returns 1 if a bit of $x$ is 1 and 0 otherwise.
Then, by definition, for all probabilistic poly-time algorithms $A$: $P(A(E_k(x)) = 1) \leq P(B(1^{n}) = 1) + \epsilon(n)$.
Clearly, if $A$ cannot determine if a bit of $E_k(x) = 1$, then by definition the $P(A(E_k(x) = (i,b)$ s.t. $x_i = b)$ requirement of computational security is satisfied. 
Further, the probability of a random bit being $1$ is $\frac{1}{2}$ so $P(B(1^{n}) = 1) = \frac{1}{2}$. 
Then our definition becomes: $P(A(E_k(x)) = 1) \leq \frac{1}{2} + \epsilon(n)$ and semantic security satisfies computational security.

\subsection*{b}
Let $G$ be a pseudo-random generator mapping $\left\{0, 1\right\}^n$ to $\left\{0, 1\right\}^m$ , and (E,D) be an encryption scheme where $E_k(x) = x \oplus G(k)$ and $D_k(y) = y \oplus G(k)$. Let $A$ be any probabilistic poly-time algorithm and $x \in_R X_n,k \in_R \left\{0,1\right\}^n$.
\\\\
Consider the algorithm $A(E_{U_n}(0^m))$. Since $G$ is a pseudo-random generator, decrypting $E_k(x)$ will require guessing each bit of $x$. The probability of guessing 1 of $2^{|x|}$ different bitstrings is uniformly distributed, so an optimal guessing strategy would be guessing all 0's since it is just as likely to be correct as any other guess. Thus $P[A(E_k(x))=f(x)] \le P[A(E_{U_n}(0^m)) = f(x) + \epsilon(n)$ and this encryption scheme is semantically secure by definition.

\subsection*{c}
For the definitions to be equal, it must be proved in both directions.\\\\
The general definition to the special case is trivial, since the special case is an assignment to the general definition, namely that $X_n$ is specifically the uniform distribution over a pair of strings $x_0^n, x_1^n$, and $f$ is the function that maps $x_0^n$ to $0$ and $x_1^n$ to $1$.\\\\
Therefore the bulk of the proof is proving that the specific random variable and the specific function are sufficient for the general definition of semantic security. Since all possible $B$ functions are equally powerful, let $B = A(E_{U_n}(0^m))$.

\end{document}
