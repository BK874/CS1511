\documentclass[letterpaper,notitlepage,twoside]{article}

% Basic imports, increase margins...
\usepackage[margin=0.75in]{geometry}
\usepackage{amssymb}
\usepackage{amsmath}

% Finite State Machine stuff
\usepackage{pgf}
\usepackage{tikz}
\usetikzlibrary{arrows,automata}

% Format tables nicely
\usepackage[latin1]{inputenc}
\usepackage{array}
\usepackage{booktabs}
\setlength{\heavyrulewidth}{1.5pt}
\setlength{\abovetopsep}{4pt}

\usepackage{amsfonts} 
\usepackage{amssymb}
\usepackage{amsmath,amsthm}
\usepackage{enumerate}

\renewcommand{\implies}{\Rightarrow} % redefine command "implies"  
\renewcommand{\iff}{\Leftrightarrow} % double arrow
\newcommand{\maps}{\rightarrow} % define command "map" 
\newcommand{\union}{\cup}
\newcommand{\intersect}{\cap}
\newcommand{\N}{\mathbb{N}} % natural number 
\newcommand{\Q}{\mathbb{Q}} % rational number 
\newcommand{\R}{\mathbb{R}} % real number 
\newcommand{\Z}{\mathbb{Z}} % integers 
\newcommand\tab[1][1cm]{\hspace*{#1}} %\tab command

% Add more packages that you use here...

\begin{document}
\title{Homework 7}
\author{Joe Baker, Brett Schreiber, Brian Knotten}
\maketitle

\section*{9}
The video details Leonard Susskind's argument for what happens to the information of objects that fall into black holes that ended the "Black Hole War" between he and Stephen Hawking. Leonard explains that, in string theory, elementary particles are viewed as having "vibrations on top of vibrations" that go faster the farther away they are from the particle. Because the gravity of a black hole warps time and space, objects appear (from an outside point of view) to slow down as they approach the event horizon, which allows an increasing amount of the vibrations to be seen. The vibrations appear as a 2D "scrambled mess" (referred to as a hologram) on the event horizon that represents the 3D object now at the center of the black hole. Therefore the information of an object that falls into a black hole is not lost: it is present both at the central mass and at the shimmering hologram at the event horizon. This holographic theory also applies to the entire universe as one can view reality both as the 3D world we perceive and as a flat, holographic film that exists at the edge of the universe.

\section*{10}

\subsection*{a}
\begin{enumerate}[(i)]

\item The entropy of  $X$ is:\\
$\begin{aligned}
H(X) &= P(x = 0) \cdot log(\frac{1}{P(x = 0)}) + P(x = 1) \cdot log(\frac{1}{P(x = 1)}) \\
&= \frac{1}{3} \cdot 1.58496 + \frac{2}{3} \cdot 0.58496 \\
&= 0.918293 \\ 
\end{aligned}$

\item The probability distribution $Y$ is: $P(y = 0) = \frac{11}{20}$ and $P(y = 1) = \frac{9}{20}$ \\

\item The entropy of $Y$ is: \\
$\begin{aligned}
H(Y) &= P(y = 0) \cdot log(\frac{1}{P(y = 0)}) + P(y = 1) \cdot log(\frac{1}{P(y = 1)}) \\
&= \frac{11}{20} \cdot 0.862496 + \frac{9}{20} \cdot 1.152003 \\
&= 0.992771 \\
\end{aligned}$

\item The conditional entropy $H(X | Y)$ is: \\
$\begin{aligned}
H(X|Y) &= P(0,0) \cdot log(\frac{1}{P(0|0)}) + P(0,1) \cdot log(\frac{1}{P(0|1)}) + P(1,0) \cdot log(\frac{1}{P(1|0)}) \\ 
&+ P(1,1) \cdot log(\frac{1}{P(1|1)}) \\
&= 0.3 \cdot 0.92891 + 0.033333 \cdot 3.754908 + 0.133333 \cdot 2.044396 + 0.533333 \cdot 0.044394 \\
&= 0.700097 \\
\end{aligned}$

\item The conditional entropy $H(Y | X)$ is: \\
$\begin{aligned}
H(Y|X) &= P(0,0) \cdot log(\frac{1}{P(0|0)}) + P(0,1) \cdot log(\frac{1}{P(0|1)}) + P(1,0) \cdot log(\frac{1}{P(1|0)}) \\ &+ P(1,1) \cdot log(\frac{1}{P(1|1)}) \\
&= 0.495 \cdot 0.15203 + 0.11 \cdot 2.321928 + 0.045 \cdot 3.321928 + 0.36 \cdot 0.321928 \\
&= 0.64139 \\
\end{aligned}$

\item The mutual information $I(X ; Y)$ is: \\
$\begin{aligned}
I(X;Y) &= 0.918293 - 0.700097 \\ 
&= 0.218196 \\
\end{aligned}$

\item The mutual information $I(Y ; X)$ is: \\
$\begin{aligned}
I(Y;X) &= 0.992771 - 0.64139 \\
&= 0.35181 \\
\end{aligned}$

\item

\end{enumerate}

\subsection*{b}


\end{document}
