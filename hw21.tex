\documentclass[letterpaper,notitlepage,twoside]{article}

% Basic imports, increase margins...
\usepackage[margin=0.75in]{geometry}
\usepackage{amssymb}
\usepackage{amsmath}

% Finite State Machine stuff
\usepackage{pgf}
\usepackage{tikz}
\usetikzlibrary{arrows,automata}

% Format tables nicely
\usepackage[latin1]{inputenc}
\usepackage{array}
\usepackage{booktabs}
\setlength{\heavyrulewidth}{1.5pt}
\setlength{\abovetopsep}{4pt}

\usepackage{amsfonts} 
\usepackage{amssymb}
\usepackage{amsmath,amsthm}

\renewcommand{\implies}{\Rightarrow} % redefine command "implies"  
\renewcommand{\iff}{\Leftrightarrow} % double arrow
\newcommand{\maps}{\rightarrow} % define command "map" 
\newcommand{\union}{\cup}
\newcommand{\intersect}{\cap}
\newcommand{\N}{\mathbb{N}} % natural number 
\newcommand{\Q}{\mathbb{Q}} % rational number 
\newcommand{\R}{\mathbb{R}} % real number 
\newcommand{\Z}{\mathbb{Z}} % integers 
\newcommand\tab[1][1cm]{\hspace*{#1}} %\tab command

% Add more packages that you use here...

\begin{document}
\title{Homework 21}
\author{Joe Baker, Brett Schreiber, Brian Knotten}
\maketitle

\section*{35}

Determining whether a given set $S$ is a certain size can be proven using a MA proof system with a constant number of steps as follows: \\\\

The trick is that Arthur will randomly generate a finite number of hash functions. Let $h_i(x) = r_i$ denote a given hash function and its resultant string. Let $n$ be the number of hash functions Arthur generates. Along with these hash functions, Arthur will generate $n$ random strings. \\\\

In the first round, Arthur sends over the $n$ hash functions and $n$ strings. If $S$ is in fact a sufficiently large set, then Merlin should be able to find a certain number of strings $x$ in $S$ that resolve to a portion of the $n$ random strings. Merlin is powerful enough to try every string in $S$ against each of the $n$ hash functions. Let this number of strings be $m$ such that $m < n$. If $S$ is not a large enough set to produce $m$ correct strings, then Merlin tries to lie to Arthur by sending over meaningless strings that don't resolve any of the hash functions or are not in $S$. Merlin passes these $m$ strings back to Arthur. \\\\

Arthur can then check these $m$ strings and resolve that in fact there are at least $m$ strings in the language or not. At this point Arthur can verify or falsify that $S$ is sufficiently large. \\\\

The trick is to find clever values of $m$ and $n$ such that they can be satisfied if $S$ is sufficiently large (that is, twice as large as the lower bound). \\\\

\end{document}
