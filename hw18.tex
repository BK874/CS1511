\documentclass[letterpaper,notitlepage,twoside]{article}

% Basic imports, increase margins...
\usepackage[margin=0.75in]{geometry}
\usepackage{amssymb}
\usepackage{amsmath}

% Finite State Machine stuff
\usepackage{pgf}
\usepackage{tikz}
\usetikzlibrary{arrows,automata}

% Format tables nicely
\usepackage[latin1]{inputenc}
\usepackage{array}
\usepackage{booktabs}
\setlength{\heavyrulewidth}{1.5pt}
\setlength{\abovetopsep}{4pt}

\usepackage{amsfonts} 
\usepackage{amssymb}
\usepackage{amsmath,amsthm}

\renewcommand{\implies}{\Rightarrow} % redefine command "implies"  
\renewcommand{\iff}{\Leftrightarrow} % double arrow
\newcommand{\maps}{\rightarrow} % define command "map" 
\newcommand{\union}{\cup}
\newcommand{\intersect}{\cap}
\newcommand{\N}{\mathbb{N}} % natural number 
\newcommand{\Q}{\mathbb{Q}} % rational number 
\newcommand{\R}{\mathbb{R}} % real number 
\newcommand{\Z}{\mathbb{Z}} % integers 
\newcommand\tab[1][1cm]{\hspace*{#1}} %\tab command

% Add more packages that you use here...

\begin{document}
\title{Homework 18}
\author{Joe Baker, Brett Schreiber, Brian Knotten}
\maketitle

\section*{29}
$BP NP$ is the set of languages which can probablistically reduce to 3SAT with a probability of $\frac{2}{3}$.
Let $L$ be a language in $BP NP$.

$L$ can be probablistically reduced to 3SAT with probability of failure $\frac{1}{2^n}$ using a circuit $C$, this is possible, because $C$ can be produced by running a reducer TM $R$ enough times such that the probability of failing to reduce is $\frac{1}{2^n}$.

Let the randomized reduction to 3SAT be conducted by a random bitstring of length $m$, which is functionally dependent on $n$. There exist $2^m$ possible random reductions, which produces a 3SAT instance that accurately represents

There are $2^m$ possible reductions given $m$. For any input $x$ of size $n$ bits, there are at most $\frac{2^m}{2^{n + 1}}$ reductions that are not correct. By the union bound over all inputs, there are at most ${2^n} * \frac{2^m}{2^{n + 1}} = \frac{2^m}{2}$ reductions which are not correct out of the total $2^m$ reductions. By the probablistic method, there must be at least one reduction which is correct for all inputs. Hard-code this string $m$, along with the NP/poly machine that solves 3SAT, and you will have a circuit that solves $L$. Therefore, $BP NP \subseteq NP/poly$.
\section*{30}
Let $L$ be a language in BPL. This means that a TM for $L$ correctly decides with $\frac{2}{3}$ probability on input $x$ with a logarithmic amount of space.

Let $N$ be a TM in P with the following behavior: \\
On input $<M, x>$, where $M$ is a BPL Turing Machine: \\
Let $M$ have a probability $\frac{1}{2}$ to take one of two transitions on each configuration. \\
Let $m$ be the maximum number of steps $M$ takes on input $x$. \\
Enumerate $2^m$ different bitstrings such that each bit represents a choice at each transition. \\
Count the total number of accepting and rejecting outcomes, and accept if the majority of the random bitstrings cause $M$ to accept. Otherwise, reject. \\

Since the language of $M \in$ BPL, the majority of probablistic outcomes will result in $M$ accepting if $x$ is in the language. So $N$, a P TM, can deterministically decide membership for a probablistic TM $M$.
\end{document}
